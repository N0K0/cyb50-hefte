\chapter[Foreningsfesten]{Foreningsfesten – et tilbakeblikk}

\author{Skrevet av Joshi}

Til de som husker student- og foreningsmiljøet ved instituttet før vi flyttet inn i Ole-Johan Dahls hus, så kan man ikke annet enn å se tilbake på et høyst begivenhetsrikt tiår. Plutselig var alle instituttets studenter samlet på ett sted, og det tok ikke lang tid i nytt bygg før man så ringvirkningene av at foreningene nå satt med mer kontorplass, mer tillit og mer ansvar enn noen gang før. Det ble blåst liv i gamle tradisjoner og det lagt grobunn for nye. Noen av de første større arrangementene vi hadde i dette bygget oppstod i en salig kombinasjon av begrenset erfaring og ustanselig eksperimentering i nye lokaler, men brikkene falt raskt på plass, og det hele kulminerte i noen av instituttets – og universitetets – største og beste fester allerede det første året i nytt bygg.

Året var 2011. Det var fortsatt god plass på foreningskontoret. Kjøkkenet var ryddet og kjøleskapet ble brukt slik kjøleskap er ment å brukes. Det var ikke slått hull i veggene og det fantes ingen minner etter FK-nachspiel i sofaene. Det luktet rent. Og CYB sine medlemmer var knapt å se på FK da de hadde nok med å legge grunnlaget for den bautaen som Escape har blitt den dag i dag. På den tiden var folk stort sett medlem i én forening – og det var svært begrenset med dialog mellom foreningene. Det var på tide å samle alle ildsjeler ved instituttet som bidro til det stadig voksende foreningsmiljøet, og det var på tide å bygge bro mellom foreningene.

Ideen om Foreningsfesten ble presentert som det første større samarbeidet mellom de fire store foreningene på den tiden: CYB, dagen@ifi, FUI og Navet. Og idéen lagt frem for FU var nokså ukomplisert: dette skulle være en anti-seremoniell, demokratisk og prisgunstig kveld i regi av representanter fra de ulike foreningene. Anti-seremoniell som i at man pynter seg så godt det lar seg gjøre; demokratisk i form av at foreningene selv velger hvem som bør inviteres; og økonomisk forsvarlig slik at en student som prioriterte foreningsarbeid over deltidsjobb også skulle kunne føle at de hadde råd til å delta. I anledning CYBs 42-årsjubileum ble det organisert en jubileumsgalla oktober 2011 og i håp om at dette skulle bli en tradisjon ble den første Foreningsfesten planlagt til våren 2012. Noen av de første møtene om Foreningsfesten i 2011-2012 ble avholdt i utdødde PING sine foreningslokaler i Veilaben – som også forduftet med tiden.

Den første Foreningsfesten ble avholdt i kantina fredag 27. april 2012 og ble litt som en kollektiv blinddate mellom de rundt 75 foreningsaktive studentene fra CYB, dagen@ifi, FUI, Mikro, Navet, PI:SK, PING, Robotica og Verdande som valgte å møte opp. I tillegg ble ildsjeler fra Sonen, instituttets TG-crew og MNSU invitert. Få visste noe om de øvrige foreningene, og det å kjenne navnet var ikke synonymt med å vite om deres medlemmer og foreningsaktiviteter. Vi måtte sette av tid i programmet til å la alle foreningene presentere seg på scenen ovenfor hverandre, og sammenliknet med nyere utgaver av Foreningsfesten var det en helt annen fest.

Den første Foreningsfesten var også en mer beskjeden og spartansk utgave enn vi ser i dag. Vi hadde ingen scene, vi brukte mesanintrappa uten høyttaler som talestol, og det var variabel grad av energi lagt i de oppmøttes klesvalg for kvelden. Vi fikk leid billig lydutstyr fra noe som muligens var en kristen ungdomsklubb – og dette ble senere stjålet fra FK sammen med Navet sitt kamera. Vi lånte servise fra kantina og det endte med at de som arrangerte stod og vasket opp 100 tallerkener og bestikk for hånd i finkjole og dress gjennom kvelden i den lille utslagsvasken som fantes i den overoptimistiske SIO-kaffebaren hvor dagens resepsjon befinner seg. Det hører med til historien at kummen tettet seg og vi klarte å oversvømme hele resepsjonsarealet. Det ble skrevet en drikkevise til Foreningsfesten satt til Aasmund Nordstogas melodi "Det er meg det samme hvor jeg havner når jeg dør". Passe spydig, en solid dose internhumor, og på ingen måte politisk korrekt anno 2019. Den forsvant fra programmet nokså raskt sammen med et knippe sanger fra CYB sitt sanghefte.

På den mer seriøse siden ble det holdt taler om viktigheten av brobygging mellom foreningene, det ble delt ut ærespriser, og kvelden ble avsluttet med et felles gruppebilde. Dette var heldigvis nok til at den første Foreningsfesten ble en suksess, og opp gjennom årene har dette forblitt den største samlingen av ildsjeler fra foreningsmiljøet ved instituttet – inkludert deltagelse fra en hel rekke foreninger som har rukket å både oppstå og fordufte i mellomtiden. Og Foreningsfesten har vokst som arrangement i takt med den enorme utviklingen i foreningsaktiviteten: de første årene var det vanskelig å spore opp gode kandidater til å motta priser; i dag er det vanskelig å velge ut noen blant alle de gode forslagene som kommer inn.

Opp gjennom årene har Foreningsfesten også vært en arena for utprøving av nye ideer – noen mer spiselige enn andre. Minneverdige innslag unnfanget på Foreningsfesten inkluderer auksjonering av studenter, organisert nattbading i dammen, rød løper med pressevegg, støvledrikkekonkurranse, bake off, Kahoots med litt for avslørende innhold, standup, helsidekonkurranser i studentavisen, livemusikk, adjektivhistorier og en hel rekke Instagrambilder ingen vil se. Ifi-Ordenen sin første marsj ned trappa til Star-Wars' «Imperial March» på høyttaleranlegget var også noe som litt tilfeldig ble til en tradisjon på en Foreningsfest. Noen innslag etablerte seg som faste tradisjoner, mens andre forble heldigvis en engangsgreie, f.eks. FU sitt bidrag til kakekonkurransen i form av en kake laget av penger.

Sett utenfra kan nok Foreningsfesten oppfattes som både elitistisk og forfengelig ettersom ingen andre institutt eller fakultet har et tilsvarende arrangement. Men ingen andre institutt eller fakultet har et tilsvarende foreningsmiljø. Foreningsfesten har alltid forsøkt å gjenspeile den enorme frivillige innsatsen som legges ned i instituttets foreningsarbeid, og det høres kanskje ut som mimring lenger tilbake i tid enn knapt et tiår, men i foreningsøyemed var det en helt annen tidsalder. Tradisjoner oppstår og fordufter, og hvem vet hvor lenge Foreningsfesten holdes i live. Men om det skulle ta slutt en dag er det en verdig trøst at Foreningsfesten gjorde sin del av innsatsen for å bygge bro mellom foreningene og samtidig fikk være med på transformasjonen fra det foreningslivet var i det gamle bygget til det fantastiske miljøet det etter hvert har blitt i vårt nye hjem. 

En ekstra stor takk rettes til alle som har bidratt til Foreningsfesten opp gjennom årene: Christopher Culina (PING), Ole Henrik Hellenes (CYB), Marte Hesvik Frøyen (dagen@ifi), Sarah Beate Hernæs (Navet), Simon Oliver Ommundsen (Navet), Thorvald Henrik Glad Munch-Møller (FUI), Per Wessel Nore (FUI), Magnus Olden (FUI), Johanne Håøy Horn (dagen@ifi), Mathias Johan Johansen (ProgNett), Kristin Brænden (Navet), Emilie Hallgren (FUI), Ole Kristian Rosvold (dagen@ifi), Espen Wøien Olsen (Navet), Oda Sofie Dahl Eide (Navet), Martine Rolid Leonardsen (Navet), Odd-Tørres Lunde (CYB), Celina Moldestad (Navet), Nikolas Papaioannou (CYB), Adrian Helle (CYB), Andreas Nyborg Hansen (CYB).

\begin{figure}
	\includegraphics[width=\textwidth]{images/foreningsfest/forste-fellesbildet.png}
	\label{fig:forste-fellesbilde}
	\caption{Den første foreningsfesten (Foto: Marte Hesvik Frøyen)}
\end{figure}
