\chapter*{Oversikt over foreninger på huset}

Det er etterhvert mange foreninger som er eller har vært aktive på Ifi. Dette er et forsøk på å liste opp dem man vet om\footnote{Hentet fra \url{http://ordenen.ifi.uio.no/association/}}.

\begin{description}
	\item[Applitude] Foreningens formål er å skape og vedlikeholde et godt sosialt og faglig miljø for mobil apputvikling blant studenter ved Universitetet i Oslo. Foreningen skal også tilrettelegge for kontakt mellom studenter på forskjellige studieretninger, ferdighetsnivåer og årstrinn.
	\item[CYB] Cybernetisk Selskab er en sosial, faglig og kulturell forening for alle studenter tilknyttet Ifi.
	\item[dagen@ifi] Hvert år arrangeres dagen@ifi som er det største studentdrevne heldagsarrangementet ved UiO.
	\item[Defi] Designforeningen ved IFI er linjeforeningen for design, bruk og interaksjon ved UiO
	\item[Digitus] Foreningen Digitus har som formål å skape et godt faglig og sosialt miljø for studentene ved studieprogrammet Informatikk: Digital Økonomi og Ledelse, ved å arrangere sosiale og faglige arrangementer. Digitus skal tilrettelegge for at alle studenter ved kan føle en tilknytting til sitt studieprogram som interesseområde.
	\item[FUI] Fagutvalget ved Institutt for informatikk er informatikkstudentenes eget organ, og skal fungere som et bindeledd mellom studentene, instituttet og universitetet forøvrig.
	\item [FIFI] Fotball ved Institutt for Informatikk er en sosial og sportslig forening ved instituttet.
	\item [Homebrew] Homebrew skal fremme kunnskap og interesse om brygging av øl og vin blant studenter ved Institutt for Informatikk.
	\item [IFI makers] IFI makers holder til på Institutt for informatikk, for å samle såkalte makers.
	\item [IFI Rotor] Multikopterforening for studenter.
	\item [IFI Sjakk] IFI Sjakk skal ha jevnlige arrangementer med fokus på sjakk.
	\item [IFI-Avis] IFI-Avis er foreningen bak magasinet Output, som gis ut én gang i semesteret ved Institutt for Informatikk. Output ønsker å informere om det som skjer inne og rundt Ole Johan Dahls hus, underholde i lunsjen, og holde deg i loopen.
	\item [Jenteforeningen Verdande] Verdande er jenteforeningen ved instituttet. Formålet er å knytte bånd mellom jenter på instituttet og andre kvinner og jenter som jobber med eller er interesserte i informatikk.
	\item [Mikro] Nanoelektronikk, Robotikk og Digital Signalbehandling (Mikro) er studentforeningen for studenter tilknyttet Nanoelektronikk, Robotikk og Digital Signalbehandling.
	\item [Navet] Navet er bedriftskontakten ved instituttet, drevet av studentene. Hensikten med Navet er å gi studentene innblikk i IT-bransjen, samt tilrettelegge for nettverksbygging.
	\item [ProgSys] ProgSys er en studentforening ved Institutt for informatikk. Foreningens formål er å fremme det sosiale og faglige miljø ved master- og bachelorprogrammene Programering og Systemarkitektur (samt Programmering og Nettverk). Foreningen skal holde faglig relevante arrangementer for programområdets studenter og sørge for at programrommet har en atmosfære som oppfordrer til sosial og faglig mingling.
	\item [Sonen] Åpen sone for eksperimentell informatikk er et studentlaboratorium ved Institutt for informatikk, Universitetet i Oslo. Et prosjektbasert møtested for engasjerte og nysgjerrige studenter, en annerledes læringsomgivelse og et kreativt lekerom.
	\item [Toastjærn] Toastjærn er en forening som vil spre glede rundt toasting, og skape samhold og tilhørighet i studentmiljøet.
\end{description}

\subsection*{Ikke lenger aktive}

\begin{description}
	\item [IT-SLP] Programutvalget for IT: Språk, Logikk og Psykologi (IT-SLP) skal være her for deg; studenten. Vi skal sørge for at du får et godt miljø å studere i, vi skal arrangere sosiale aktiviteter og være din forbindelse til programmets ledelse og ansatte.
	\item [PI:SK] Programutvalget for Informatikk: språk og kommunikasjon (PI:SK) jobber for et bra faglig og sosialt miljø rundt studieprogrammet Informatikk: språk og kommunikasjon.
	\item [PING] Program-, Informasjons- og Nettverksteknologisk Gruppe (PING) er en studentforening ved Universitetet i Oslo. På disse sidene finner du litt generell informasjon om foreningen.
	\item [ProgNett] ProgNett er linjeforeningen for studenter ved bachelor- og masterprogrammene Programmering og nettverk
	\item [ProsIT] Linjeforeningen Profesjonsstudentene ved Institutt for Informatikk (ProsIT) er den sosiale foreningen for profesjonsstudentene ved Institutt for informatikk på Universitetet i Oslo. Vi jobber for at alle profesjonsstudentene skal ha en sosial og festlig studietid. Vi arrangerer jevnlig fester og turer, og tilbyr aktiviteter som bedriftsbesøk, idrett og leking med Lego Mindstorms. ProsIT har også ansvaret for fadderordningen for profesjonsstudentene, og vi vil gjøre vårt beste for at alle nye studenter skal få en fin start på semesteret.
	\item [Svett] Navnet Svett spiller på den populært utbredte stereotypen av informatikere. Mange ser på informatikere som svette «nerder». Vi velger å spille videre på dette med et glimt i øyet.
	\item [TekNat] Tekniske og Naturvitenskapelige anvendelser (TekNat) er en studentforening for deg som liker realfag og matematikk, og som ønsker å bidra til et faglig og sosialt miljø for disse studentene.
\end{description}
