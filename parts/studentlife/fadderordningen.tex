\chapter{Fadderstyret ved Institutt for informatikk}

\author{Skrevet av Thao Tran, leder for Fadderstyret ved Institutt for informatikk, og Arne Hassel}

Fadderordningen på Ifi har alltid vært nært tilknyttet studiemiljøet og ildsjelene som ``var med på alt''. Gjennom historien har fadderordningen vært gjennomført i regi av alt fra grupper uten nær tilknytning til noen foreninger, til grupper med nær til èn bestemt forening, til grupper med tilknytninger til flere foreninger. Det var også tendenser til mer oppsplittede fadderordninger, hvor bestemte grupperinger gjennomførte opplegg for ``sine'' studenter og ikke nødvendigvis hadde fokus på alle som skulle starte på Ifi.

I 2009 ble Cybernetisk Selskab spurt om å ta en aktiv rolle i fadderordningen, i håp om å skape rammer for en fadderordning som inkluderte alle, uavhengig av hvilken studieretningen man startet på. Denne oppfordringen takket man ja til, og frem til med 2011 tok foreningen ansvar for å samle et fadderstyret som så tok kontroll over spakene videre. Fra 2012 begynte fadderstyret å stå mer på egne bein, og i 2014 ble Fadderstyret ved Institutt for informatikk opprettet som egen forening, med egne vedtekter og gjennomføring av generalforsamlinger.

Foreningen tar for seg alt som har med de første to ukene på høstsemesteret, og bidrar også til opplegg for dem som starter på vårsemesteret. Ikke lange tiden, men det ligger mye arbeid bak opplegget. Sånn sett kan Fadderstyret ses på som det indre organ som enkelte setter pris, men andre derimot har lett for å glemme. Uansett vil dette indre organet være nødvendig, som et bindepunkt hvor hovedoppgaven vil være å knytte samt samkjøre Ifi til en fullkommen helhet.

Studiemiljøet er primærnøkkelen til oppstandelsen av fadderstyret, hvor hovedfokuset er å skape et miljø som får studenter igjennom studietiden på best mulig. Ved å fremme at Ifi er et multikulturelt sted som dekker hver enkelt students behov, bidrar det til å skape en trygghet på studiet, men også den sosiale fronten.

I flere år har fadderstyret stått for en fadderuke, som i dag blir studiestart uken, hvor hovedfokuset er å få nye studenter til å trives på de ulike programlinjene. Ved hjelp av motiverte faddere ved Ifi samarbeider vi for å skape en langvarig trivsel som kan bidra til et varmt og åpent miljø hvor alle føler seg komfortable og velkomne.
