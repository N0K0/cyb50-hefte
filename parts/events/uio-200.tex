\chapter[UiO:200]{UiO:200 - Bursdagsfeiring av UiO og offisiell åpning av OJD}

\author{Skrevet av Arne Hassel}

2. September 2011 var en merkedag for Universitetet i Oslo og Institutt for informatikk. Da feiret man både 200 år for universitet og den offisielle åpningen av instituttets nye bygg, Ole-Johan Dahls hus. Det var en dag fylt med fest og glede, med start på fakultetene hvor kransekake og cava ble konsumert, etterfulgt av konsert på Frederikkeplassen med Bigbang og venner, en utrolig åpningsfest på OJD, og avslutning på Chateau Neuf med dansing ut i de sene timer.

For mange av oss som studerte ved Ifi var spesielt åpningen av OJD en stor hendelse, så selv om UiO:200 egentlig betegner hele jubileumsåret og styret som jobbet med planleggingen av den, så brukes navnet også spesielt om festen på OJD når man snakker med dem som var i Ifi-miljøet på den tiden. Det var også en anledning for Cybernetisk Selskab til å virkelig vise muskler. Med et styre på seks personer og 105 funksjonærer klarte man å gjennomføre en fest med to konserter, standup-show, flere aktivitetsrom og ikke minst 8 barer. Og dette trengtes virkelig når man fikk besøk av 4-5000 mennesker i løpet av noen timer.

Planleggingen av festen begynte en god stund før, men startskuddet for studentene var høsten 2010, hvor UiO:200-styret introduserte planen for studentene i SVLED1900 - Studentledelse. Dette var et fag for aktive i studentforeninger, og var en arena for studenter fra alle fag på universitetet til å dele erfaringer og få opplæring i prosjektledelse og samarbeid. Blant annet var Arne Hassel og Magnus Johansen med på dette, og de ble satt i gruppen som skulle komme med et forslag på gjennomføring av festen.

Arbeidet med studentledelse ble en naturlig bro videre til planlegging av arrangementet, og etterhvert fikk man på plass et studentstyre bestående av Arne Hassel (Festgeneral), Christian-Magnus Mohn (Barsjef), Vegard Angell (Teknisk sjef), Åshild Aaen Torpe (Underholdningssjef), Martin Bore (Vaktsjef) og Torgeir Lebesbye (Personalsjef). I tillegg var Magnus Johansen en viktig samarbeidspartner, da han var Kjellermogul i Escape. Studentforeninger som Filologisk Forening, Norsk Klassisk Studentforening, Realistforeningen og Samfunnsvitenskapelig Fakultetsforening var også viktige samarbeidspartnere som drev flere av barene. Men brorparten av arbeidet og barene ble drevet av Cybernetisk Selskab, som stilte med rundt halvparten av funksjonærene som trengtes.

Det var mye som skulle på plass før den store festen, og det ble mange møter og kontakter med samarbeidspartnere. I tillegg til den sentrale festkomiteen i UiO:200 var også Teknisk Avdeling, Studentsamskipnaden i Oslo, Institutt for informatikk, Chateau Neuf Servering og Åpen sone for eksperimentell informatikk med på planleggingen, og det var bare de som var tilknyttet universitetet; i tillegg var 15 bedrifter med å gjøre alt som trengtes, som inkluderte alt fra teknisk utstyr til scene, festivaltoaletter, t-skjorter til funksjonærer, ekstra deilig inventar (vi leide inn laidbags - basically kjempestore puter som folk kunne sitte i) og ikke minst øl og annen drikke. På det sistnevnte tok man noen vågale valg, som f.eks. å kjøpe inn masse cider med jordbær og pære-smak, noe man trodde skulle være en hit etter sommeren, og som viste seg å være helt, helt feil (heldigvis kunne man returnere det meste).

En av de største stressfaktorene med planleggingen var å få rekruttert nok folk. Noe av problemet var at man ikke kom i gang med skikkelig rekruttering før i mai, og da var studentene opptatt med eksamener. Sommerferien kom og gikk, og rekrutteringen under fadderukene gikk heller ikke som ønsket. Det som ble redningen var rekrutering i form av et pyramidespill, hvor man spilte på et av de mest effektive betalingsmidlet for studenter - drikkebonger! Man designet det slik at foreninger fikk flere bonger jo flere medlemmer de fikk til å delta, og det oppsto da nærmest en konkurranse mellom foreningene om hvem som kunne rekruttere flest. Bongsystemet ble også utvidet til å belønne dem som tok på seg mange vakter, og her er det tydelig at de som jobbet i den tekniske gruppen virkelig sto i. På førsteplass av antall bonger utdelt var Teknisk sjef selv, Vegard Angell, med hele 46 bonger utdelt. Ikke langt bak kom andre fra gruppa, nemlig Sjur Hernes (som jobbet som en gud hele festen), Atle Nordland og Veronika Heimsbakk. Til sammen delte man ut 2 061 bonger, hvor 1 630 av disse var på grunn av rekruteringen.

På selve dagen viste det seg fort at folk var feststemning. På starten dannet det seg kø på broen over fra Blindern, da flere tusen mennesker begynte å strømme til fra konserten på Frederikke. Heldigvis fant også noen ut at man kunne komme inn via Nordtårnet, så strømmen delte seg greit. Det ble også en voldsom folkemengde i tredje etasje, hvor man hadde DJ Finanskrisa og fire barer. Tidligere på kvelden hadde man prøvd å promotere tredje etasje som ``Norges lengste bar'', men dette ble fort forandret til (av ukjente kunstnere) ``Norges lengste barkø''.

Kveldens høydepunkter besto av to konserter i kantina, Tôg og Oslo Ess, og standup i Simula, med Jonas Rønningen, Cecilie Steinmann Neess og Andy Taffs. Simula sleit med at ingen mikrofoner ville fungere, noe Jonas heldigvis løste ved å hente frem en megafon. Opplegget med artistene gikk heller ikke helt smertefritt, da man i entusiasmen over å vise frem det fine nye bygget hadde lagt backstage til Informatikksalen. Fin som den er så er den dessverre i femte etasje, på andre siden av bygget. Dette i kombinasjon med en fullspekket bygg hvor flere gjester gjerne ville hilse på bandet, gjorde logistikken mellom backstage og konsertområdet litt krevende. Men på tross av dette ble underholdningen godt levert, og folk storkoste seg. Et høydepunkt var at konsertene ble innledet av Morten Dæhlen, på det tidspunktet instituttleder, som avsluttet med å dra i gang We Will Rock You. Stemningen var upåklagelig, men da han begynte å dra den i gang for andre gang følte noen at det var klart for de faktiske artistene.

I tillegg til de store høydepunktene hadde man også noen mindre aktivitetsrom, hvor Smalltalk ble brukt til Loopy Tangible, et studentprosjekt hvor man lagde musikk med å dekke til hull av forskjellige farger, Java ble brukt av Sonen, som viste frem spennende prosjekter de holdt på med, og Logo ble brukt til å vise fotballkamp. Spesielt sistnevnte var populært, da det var landskamp mellom Island og Norge, og folk sto ut i gangen for å få med seg spenningen.

Drama var det heldigvis lite av, selv om man ikke klarte å unngå at noen ansatte trakk seg opp i fjerde etasje for å kose seg med noen øl og mindre fester. Dette ble raskt ryddet opp i, men ikke før noen hadde glemt igjen en pose med popcorn i en mikrobølgeovn. Denne tok fyr, men heldigvis hadde ikke røyk-sensorene blitt stilt på noe annet enn varme enda, så selv om det kom noe røyk fikk man fort ordnet opp i det. Sikkerheten hadde tatt høyde for evt avbrytelser, men det hadde ikke vært moro å måtte avbryte festen og evakuere flere tusen mennesker på grunn av en pose med popcorn…

Ellers var det en ulykke med at én person falt ned trappene og fikk et lite kutt. Her fant man ut at nærheten til sykehuset var en heldig sideeffekt med bygget, med responstid på et par minutter. Litt morsomt var det også at personen som falt var ansatt i Teknisk Avdeling, som jo hadde ansvaret for sikker tilrettelegging av lokalet.

Da kvelden nærmet seg slutten kunne man begynne å senke skuldrene. Festen hadde vært en kjempesuksess, og folk vandret enten videre til Chateau Neuf eller en av de andre uoffisielle nachspiel-stedene. Den tekniske gruppen hvilte ikke på laurbærbladene, og startet nedrigging, mens de som styrte med økonomien sørge for å telle opp penger og lage sluttrapporter. Sistnevnte viste seg å ha en liten utfordring, da flere betalingsterminaler i løpet av kvelden bestemte seg for å lage avstemmingsrapport for mindre beløp. Dette gjorde at man endte opp med godt over 100 rapporter. Når man da til slutt finner ut tre mangler, og en av de var på over 100 000 kr var ikke stemningen helt på topp. Men heldigvis ordnet det seg til slutt, og spesiell takk til arbeidet rundt økonomien går til Martin Haugland, Daniel Høgli Olufsen og Ole Henrik Hellenes.

Etterarbeidet til festen var internfester med funksjonærene (man måtte jo få anledning til å bruke bongene sine) og samle inn info til rapport (denne endte opp på 123 sider til sammen, og har forhåpentligvis vært en god ressurs for senere storfester på bygget). Det var mye god stemning, og tydelig at man var fornøyd med arbeidet som ble lagt ned. Cybernetisk Selskab hadde med det fått vist at man var en studentkjeller-forening verdig, og grunnlaget for videre arbeid som instituttforening var lagt.
