\chapter{Whisky-seminar}

\author{Skrevet av Torgeir Lebesbye}

Et whisky-seminar er en whisky-smaking ikke ulikt en vin- eller ølsmaking. Med et seminar hvert semesteret siden oppstart våren 2009 har det blitt den eldste tradisjonen i Cybernetisk Selskab man fortsatt gjennomfører foruten generalforsamling. Samtlige seminarer har blitt holdt av skotten Chris Maile.

Normalt smaker man på fem whiskyer som serveres i glass konstruert for å smake og lukte på whiskyer, såkalte \textit{nosing glass}, som hvert inneholder omkring 1.9 cl whisky. Temaet for et seminar kan være whiskyer fra samme destilleri med ulik lagringstid (en \textit{vertikal smaking}), whiskyer lagret på samme fattype fra ulike destillerier, eller hvordan whiskyer kan benyttes som en dessertvin med ost eller sjokolade.

De første whisky-seminarene ble arrangert av mangeårige styremedlem Martin Lilleeng Sætra. Til å begynne med ble de holdt i VIP-kantina i det daværende Informatikkbygget (senere omdøpt til Kristen Nygaards hus). Etter åpningen av Ole-Johan Dahls hus i 2011 har de vært avholdt i studentkjelleren Escape.

Seminaret i mars 2010 tok for seg fattypens innvirkning på maltwhisky fra Glenmorangie, et destilleri nord på det skotske fastlandet kategorisert som et Highland-destilleri. Det var en klar inspirasjonskilde til da man som fersk kjellerpubforening valgte Glenmorangie The Original til sin representasjonsdrikke.

Chris ble utnevnt til æresmedlem av CYBs generalforsamling høsten 2013, en utmerkelse han passende nok fikk overrukket på whisky-seminaret våren 2014.