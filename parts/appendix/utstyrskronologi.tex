\chapter*{Utstyrskronologi}

\begin{tabularx}{\textwidth}{ll}
	1977: & Første mikromaskin\slash PC: MYCRO-1. Innkjøpt av Per Ofstad for Musicus-prosjektet \\
	& Egen terminalstue for lavere grad ved EDB-senteret med \\
	& 15 skjermterminaler (Behive) \\
	1981: & Første arbeidsstasjon (PERQ) med rastergrafisk skjerm \\
	1982: & Egen stormaskin (DEC 20) for undervisning. Administrert av EDB-senteret \\
	& Første bruker av Tandbergs moderne terminal Universitetets første VAX 11/780 til ansatte \\
	& Første Berkeley UNIX i Norge \\
	& Tilkoblet Internett \\
	& Egen terminalstue med mikromaskiner (10 stk ALTOS) \\
	1985: & Mikro VAX med x-windows (installasjon nr. 3 på verdensbasis, utenfor MIT) \\
	1987: & Europas største installasjon av distribuerte systemløsninger basert på SUN-utstyr (både for lavere grad og til arbeids- stasjoner for hovedfag/ansatte) \\
\end{tabularx}