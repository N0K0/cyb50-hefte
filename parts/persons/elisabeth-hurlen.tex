\chapter[Elisabeth Hurlen]{Elisabeth Hurlen – Ifis ``mor'' i krevende tider}

\author{Skrevet av Narve Trædal}

\begin{figure}
	\includegraphics[width=\textwidth]{images/elisabeth-hurlen/profil.png}
	\label{fig:elisabeth-hurlen}
\end{figure}

Elisabeth Hurlen, eller Beth, som alle kalte henne, var ansatt som leder av administrasjonen ved Ifi fra før instituttet ble opprettet i 1977, og fram til hun ble 70 år, i 1998.

Hun hadde hovedfag i kjemi fra 1955, og var gift med professor Tor Hurlen ved Kjemisk institutt. Begge døtrene deres tok hovedfag ved Ifi. Hun ble første ansatt i halv stilling, og det var hun, og en eldre kontorsekretær, som var de eneste administrasjonsansatte, ved et fag med eksplosiv studentvekst, og der tallet på undervisere og veiledere - og undervisningsrom - alltid var på etterskudd. Og det skulle etableres en stor stab av timebetalte studenter; gruppelærere og terminalvakter. I tillegg provisoriske tilleggsarealer, som Brakke I, der hvor Helga Engs hus nå ligger, under instituttets opphold i Fysikkbygningen, og senere ``Brakka'', nord for Informatikkbygninen, der MiNaLaben nå ligger. Før oblig-levering kunne det være lange køer utenfor disse lokalene, og det var alltid mange studenter der, natt og dag, og også i alle helger, påske og jul.

Hennes første ti år var således preget av nærmest konstante flytteprosesser: Først fra Matematikkbygningen til Fysikkbygningen, og deretter til Informatikkbygningen i Gaustadbekkdalen. Og hvert sted var det bytte av undervisningrom, kontorer, og ikke minst, etter flyttingen til Informatikkbygningen ombygginger. Hovedfagslesesaler, som det i utgangpunktet var en del av, måtte etter hvert tas i bruk til den voksende staben. Det kunne ha utløst konflikter, men Beths utpregede samarbeidsevner hindret eventuelle tilløp til slikt. Studentene forsto også at det ikke var noen vei utenom, dersom de skulle få ansatt de lærerne og det it-driftspersonale de trengte . Instituttet fikk etter hvert mer saksbehandlerhjelp, i form av forværelsestjeneste, studentadministrasjon, økonomi- og peronsalfunksjon. For Beth stilte det større krav om mer lederansvar, med rettledning og opplæring. Det taklet hun godt. Nøyaktigheit og humør var stikkord. Huns samarbeidet også hele tiden vennlig og konfliktfritt med administrasjonen på fakultetsnivået.

I de første årene etter hun ble ansatt, var det henne studentene kom til med det de måtte ha av spørsmål knyttet til studiene og studiehverdagen sin. Etter at administrasjonen vokste, og i tråd med det skjedde en foryngelses- og kompetanseheving, hadde hun mindre direkte kontakt med studentene. I de siste 15 årene ble hun nok oppfattet mer som ``mor'' av de ansatte enn av studentene. De forholdt seg mer til ``sine'' saksbehandlere, og til vitenskapelig ansatte i ulike verv; særlig leder og medlemmer av Undervisningsutvalget.
