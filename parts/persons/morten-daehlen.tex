\chapter[Morten Dæhlen]{Morten Dæhlen – strateg og studentvenn i viktige tider}

\author{Skrevet av Narve Trædal}

\begin{figure}
	\includegraphics[width=\textwidth]{images/morten-daehlen/profil.png}
	\label{fig:morten-daehlen}
\end{figure}

Morten Dæhlen ble i 2005 valgt instituttets første instituttleder, etter at instituttbestyrervervet var blitt avskaffet gjennom kvalitetsreformen og tilhørende endringer i universitetsloven. Forskjellen på en bestyrer og en leder er at lederen fikk større fullmakter, og antallet saker som krever styrebehandling blir færre. Han ble gjenvalgt for en ny periode i 2009. I 2012 ble han valgt til dekan ved Mat.nat.-fakultetet. Der ble han gjenvalgt i 2016, og er således midt i sin andre dekanperiode.

Han tok cand.scient.-eksamen ved Ifi i 1985 i fagfeltet numerisk analyse. Ved siden av hovedoppgaven hadde han jobb ved SINTEF, og han tok doktorgraden som ansatt der i 1989. I de neste femten årene kombinerte han en akademisk karriere og flere sentrale verv i forskningssektoren. Han ble ansatt i II-stilling ved Ifi i 1990, og ble full professor i 1998. Han hadde imidlertid flere permisjoner fra stillingen; først for å lede etableringen av SINTEFs MiNaLab, deretter for å være direktør for naturvitenskap og teknologi i Norges forskningsråd i 1999. Da Simula-senteret ble etablert i 2001, tiltrådte han lederstillingen der. Han kom tilbake full stilling som professor i 2004 og ble valgt til instituttleder samme høst, i konkurranse med instituttbestyrer Jens Kaasbøll, som ønsket gjenvalg. I 2009 ble han gjenvalgt, uten motkandidat.

I tillegg til hans personlige egenskaper, har Ifi således også dratt nytte av hans erfaringer fra i den eksterne forskningssektoren og forskningspolitisk arbeid i de stillingene han har hatt der. Han fremste instituttstrategiske sak i hans første lederperiode var satsing på forskerutdanningen. Tallet på stipendiater og avlagte doktorgrader økte sterkt. Organisatorisk var perioden preget av en styrking av forskningsgruppeleder-møtet, som ble etablert som et regulært møtende forum, stort sett hver andre uke, der de diskusjonene som tidligere hadde preget instituttstyret, for en stor del ble gjennomført. Dette har fortsatt etterpå. I dette forumet er ikke studentene representert. For å bøte på det, ble det avholdt uformelle møter mellom instituttleder og representanter for FUI og øvrige studentforeninger for å drøfte sentrale studentsaker. Næringslivskontakt og innovasjonsaktivitet står hans hjerte nært, og han støttet opp om etableringen av studentforeningene dagen@ifi, som så dagens lys høsten før han tiltrådte, og Navet i 2007. Han arbeidet også for å styrke Ifis aktiviteter i Innovasjonssenteret i Forskningsparken.

Ole-Johan Dahls hus var sentralt i hele hans instituttledertid. I planleggings- og byggeprosessen var han genuint opptatt av studentenes plassbehov. Hans visjon om at Ole-Johan Dahls hus også i stor grad skulle være preget av bachelorstudentene, og han så viktigheten av å la studentene slippe til i interiørplanleggingen av de 28 000 kvadratmeterne. 

Han var også en sterk pådriver for egne lokaler for studentforeningene, og en egen studentkjeller, Escape, drevet av og for, informatikkstudentene. Han har etter evne holdt ved like sosial kontakt med studentene, og stikker av og til innom Escape for en øl, også etter at han er blitt dekan ved MN-fakultetet. Dette er noe CYB har satt veldig pris på, og høsten 2013 ble han tildelt CYBs æresmedlemskap.

Etter at huset var overlevert, arbeidet han for at den rikholdige inventarbevilgningen blant annet skulle brukes til en sterk opprustning av laboratoriefasilitetene, først og fremst innen robotikk og digitalteknikk.

At en instituttleder blir dekan er ikke alltid en ubetinget fordel for hans hjemme-institutt, som av og til føler seg som offer for at dekanen skal vise sin ``upartiskhet''. Slik har det ikke vært i Morten Dæhlens tilfelle. Ifi har fått sin del av fakultetets ressurser, og nye initiativer fra instituttet blir tatt
godt i mot og vel ivaretatt på fakultetsnivået.
