\chapter[Rolf Bjerknes]{Rolf Bjerknes - CYBs første æresmedlem}

\author{Skrevet av Narve Trædal}

Rolf Bjerknes ble født i 1927. Han gikk bort vinteren 2017, nesten nitti år gammel. Han ble ansatt som førsteamanuensis ved Fysisk institutt i 1971, etter 12 år ved SI (nåværende SINTEF) og var en del av kybernetikkmiljøet som flyttet over til det nystartede Institutt for informatikk i 1977. Der var han
aktiv frem til 1997, da han gikk av etter fylte 70 år.

I sin Ifi-gjerning var Rolf i alle år svært opptatt av studentene og undervisningen. På minnesiden som ble opprettet i forbindelse med hans bortgang, står følgende formuleringer som sier mye om hvordan han ble betraktet av studentene.

``Rolf Bjerknes har vært informatikkstudentenes omsorgsfulle bestefar. Det skyldes nok også at vi kom ham nærmere enn de fleste andre lærerne, siden han var undervisningsleder for høyere grad. Bjerknes var vårt kontaktpunkt mens vi ventet og ventet for å komme videre til hovedfag. Han var tilgjengelig for oss, når problemene tårnet seg opp, og fikk avhjulpet situasjonen ved å skaffe eksterne veiledere. Mange utførte hovedfagsoppgavene ved institusjoner utenfor UiO.'' Forfatter er ukjent, men sitatet beskriver situasjonen i første halvdel av søttiårene, der det var lange køer for å ble tatt opp.

Og videre, fra Steinar Kjærnsrød, tidligere leder av IT-driftsavdelingen: ``Kjære Rolf, jeg hadde den glede av å ha deg som veileder på hovedfag og ellers som foreleser på flere kurs på lavere grad og hovedfag. Du var en varm person med faderlig omtanke for studentene dine, og hadde et stort engasjement for faget og med en egen evne til å formidle hva Laplace- og Fourier transformer og Kalmanfiltre kunne brukes til i praksis :-) Ikke minst husker jeg hvordan du fortalte om de komplekse egenverdiene du hadde funnet i et eller annet kloakkrenseanlegg du hadde studert, veldig morsomt. Jeg vil alltid minnes studietiden på IfI med stor glede, og ikke minst det tette forholdet vi hadde mellom studenter og forelesere og som du var et eksempel på. Du fortjener en stor plass i historien til Institutt for informatikk, hvil i fred.''

Tradisjonell akademisk forskning var kanskje ikke hans sterkeste side, selv om han fullførte sin doktorgrad i 1977, og publiserte enkelte artikler også etter det. Men det som var hans store interesse på publiseringssiden, var populærvitenskaplig kobling av naturvitenskap\slash fysikk og litterære uttrykk, i form av ``å gå på innsiden av'' eventyr og dikt, og produsere ``epistler og anekdoter'', som han kalte det.

Rolf var en sentral person også i CYB i mange år, blant annet som pådriver for at CYB skulle være med på Servomøtene i Trondheim. Det var tradisjon at han holdt foredrag etter generalforsamling, noe gamle cybbere snakker om den dag i dag. Akkurat når han fikk ærespris ser dessverre ut til å ha gått i den kollektive glemmeboken\footnote{Ut fra arkiver vet vi at han fikk æresprisen før 1991}, men at han var den første til å motta den er det ingen tvil om. Han var en kjent og kjær skikkelse for mange cybbere opp gjennom tiden, og på instituttet forøvrig, og han vil savnes.

