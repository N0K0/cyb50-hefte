\chapter[Kristen Nygaard]{Kristen Nygaards teknologiske konstruksjon av arbeidsplassdemokrati}

\author{Skrevet av Egil Øvrelid}

\emph{Denne teksten ble original skrevet for studenttidsskriften Index 5. mai 2016. Vi har tatt den med i denne boken til glede for nye lesere.}

% TODO: Flytt til delen om kilder
% https://issuu.com/ifiindex/docs/utgave_2_2016_web - side 8-11

Kristen Nygaard døde i 2002, 76 år gammel, men arven etter hans arbeid lever fortsatt. Den kan ses blant annet i det brede fokuset innføringen av kliniske IT-systemer har i dagens Helsevesen. Det store programmet ``Digital fornying'' i Helse Sør-Øst, som har en prislapp på 6 milliarder i perioden 2013-2020, handler blant annet om utvikling og standardisering av kliniske applikasjoner. Kravspesifikasjonene som sendes ut på anbud er utarbeidet i tett samspill med en rekke klinikergrupper. Brukernes aktive deltagelse er en selvfølge. Denne formen for deltagende utvikling har sin kilde i det vi kan kalle den ``skandinaviske modellen for systemutvikling'' som har sitt opphav i Nygaards og Ole-Johan Dahls arbeid etter krigen og Nygaards og Olav Terje Bergos Jern- og Metallprosjekt sammen med Fagforeningen på begynnelsen av 1970-tallet. Arbeidet ledet allerede tidlig i 1970-årene til at det ble inngått dataavtaler og oppnevnt datatillitsvalgte i arbeidslivet.

\section{Operasjonsanalysen}

Kristen Nygaards karriere startet på Forsvarets forskningsinstitutt (FFI) rett etter krigen. Han jobbet med prosjekter knyttet til modernisering av Forsvaret, som var tett knyttet til gjenoppbygningen av landet og industrien etter 5 år under okkupasjon. Utover 1940- og 50-tallet var Norge langt fremme både innen kjernekraft og produksjon av militærteknologi, og produksjonsmodellene herfra ble gjeldende også for annen industri. Nygaards engasjement og kunnskap vokste frem her, og det var flere elementer som påvirket hans virke frem til 1975.

Først Operasjonsanalysen som Nygaard brukte aktivt i sitt arbeid både på FFI og Norsk Regnesentral. Operasjonsanalysen (OA) er en matematisk kvantifiserbar vitenskap som anvendes for å finne det mest effektive samspillet mellom militære teknologier som fly og militært materiell i krigføringen. Operasjonsanalysen viste seg svært effektiv under andre verdenskrig. Simulering ble brukt for å modellere kommunikasjonsstrømmen mellom komponentene i den militære teknologien, og Nygaard videreutviklet operasjonsanalysens virkeområde ved å integrere soldatene tettere inn i eksperimentene, samtidig som han deltok selv. Systemanalyse er en annen retning innen OA, men dens fokus på økonomi forskjøv beregningstyngden over på en ledelsesdiskurs som handlet om å velge det mest lønnsomme, ikke lenger det vitenskapelig riktige. Kristen Nygaard kunne ikke aksepte dette. Det skiftende fokuset fra grunnivået der soldatene og teknologien opererer til Systemanalyse der økonomi og ledelse dominerer ble for mye å svelge for Nygaard. Han sa derfor opp hos FFI og gikk til Norsk Regnesentral i 1960.

\section{Arbeiderne i fokus}

Deretter er Aksjonsforskningen til Tavistock-skolen tilegnet fra gruvene i Nord-England på 1950 tallet en viktig inspirasjonskilde i Nygaards arbeid. Tavistock ble opprettet like etter første verdenskrig, og ble utvidet med ``Institute of Human relations'' i 1947, der samfunns- og arbeidsforhold sto sentralt. Forskningen til Tavistock gikk ut på å dokumentere problemene som oppsto i overgangen fra en autonom modell med selvstyrte små team, til en omfattende oppdeling av arbeidet i ulike prosesser, og med flere skift. Det viste seg at effektiviteten gikk ned, og at arbeiderne tok mindre ansvar for helheten i arbeidet. Den sosiotekniske systemforskningen har sitt opphav her, men Tavistocks ``idealtype'' med små selvstyrte team som ivaretok både nærsamfunn, arbeidet og arbeiderne, skalerte dårlig i den nye samfunnsøkonomien basert på stordrift, masseproduksjon og spesialisering. Dette ble inspirasjon for et tilsvarende prosjekt i Norge. Dette ble støttet av NAF, Jern- og Metall og den norske Stat, og det norske arbeidslivet ble sett på som spesielt egnet for slike forsøk. Målet med det norske prosjektet var å ``forbedre betingelsene for personlig medvirkning i den konkrete arbeidssituasjonen med sikte på å utløse menneskelige ressurser.'' Gjennom rotering på arbeidsoppgaver skulle arbeiderne få sterkere eierskap og friere utfoldelse på arbeidsplassen. Prinsippene fra prosessene i gruvene i Tavistock ble videreført, men tilpasset dem til den moderne industrien. Imidlertid var den strategiske og organisatoriske planleggingen i bedriften fortsatt i ledelsens vold.

Dette norske prosjektet var bakgrunnen for at Nygard og Bergo startet et prosjekt sammen med Jan Balstad fra Jern- og Metall. De hadde som eksplisitt forutsetning at samarbeidsprosjektene til Thorsrud og Emery ikke gikk langt nok i prosessforbedringen, ``at medvirkningen skjedde på et for sent tidspunkt i teknologiutviklingen, og at all kunnsakpsutvikling skjedde på ledelsens og forskernes premisser''. Nygaard var krystallklar: Arbeidstakerne måtte gis dypere i innsikt i bedriftsledelse og styring i tillegg til produksjon.

\section{SIMULA}

Kristen Nygaard var først og fremst informatiker og programmerer, og ble etter hvert sterkt drevet av objektorientert tenkning. Gjennom erfaringene med simulering fra krigen og hvordan ulike komponenter (inkludert soldaten) kan forstås som objekter i systemet, lagde Nygaard og Dahl SIMULA, verdens første objektorienterte programmeringsspråk. SIMULA ble et pedagogisk språk som muliggjorde en helhetlig systemutviklingsprosess der arbeiderne kunne delta fra spesifikasjon og planlegging og helt til innføringen av systemet i organisasjonen. I SIMULA fikk dataelementene egenskaper, og ble således dynamiske representanter i systemet for verden utenfor. Den grunnleggende endringen besto i at arbeidernes systemverden ble satt i sentrum på en helt annen måte. Jern og Metall-prosjektet tok inn i seg alle disse strømningene i en kraftfull cocktail som skulle skape en brukerstyrt teknologisk sfære som dynamisk kunne tilpasses og brukes i enhver industrisammenheng.

Vi har sett noe av løsningen til Nygaard, men hvilket samfunnsproblem var det han forsøkte å løse?

Det moderne industrisystemet som vokste frem etter andre verdenskrig var basert på sterk statlig deltagelse i industri- og samfunnsbyggingen. Det var i utgangspunktet lagt opp demokratisk, men visse krefter trakk det bort fra fokus på arbeidstakerens teknologiske utvikling, og isteden mot økonomisk eller teknokratisk optimalisering. Industrisystemet var meget komplekst, basert på teknologisk og økonomisk utvikling, og omfattende kunnskap var nødvendig for å styre det. Universitetene ble den sentrale institusjonen, og utdanning den sentrale faktoren for å bli politiker og industrileder. Den politiske og industrielle ledelsen var basert på utdanning og kunnskap fra universitetene, samtidig som arbeiderklassen havnet i bakleksa. Skillet mellom utdannede og ikke-utdannede truet balansen og den demokratiske deltagelsen i samfunnet. Et fundamentalt problem med den ledelsesorienterte og tidvis teknokratiske diskursen var at den førte til ``dekvalifisering'' av arbeidstakeren, der de som jobbet på gulvet verken hadde kunnskap eller forutsetninger til å forstå hvordan systemet fungerte. Arbeidstakeren havnet i teknologiens vold, og ble fratatt alle menneskelige egenskaper i arbeidsutføringen.

Her gir objektorienteringen arbeidstakerne et språk de kunne anvende til å kommunisere de sosiale perspektivene som skulle oversettes til teknologiske interaksjoner i systemet. Når arbeiderne selv er med å bestemme egenskapene til sine objekter i systemet påvirker de direkte styringen av systemet, fordi objektene er deler av et system som danner grunnlag for de avgjørelser som fattes av de som styrer. SIMULA skulle således bidra til en ``rekvalifisering'' av menneskelige egenskaper. En forflytning av industriell kapasitet til informasjonsteknologisk innsikt.

Kristen Nygaard ville nok vært ambivalent til de store Helse-IT prosjektene som pågår i Norge nå der sentralisering går foran desentralisering, der økonomiske perspektiver settes i forgrunnen, og der den kliniske ekspertise ofte må konkurrere mot ledelse og økonomi. På den andre side handler også det moderne helsevesenet om smarte pasienter. Et system der pasienten blir stående litt med ``lua i hånda'', prisgitt uoversiktlige maktsystemer, er ikke lenger et moderne system. Akkurat som industriarbeidere hadde påvirkningskraft i 1970-årene burde pasientene også være objekter med påvirkningskraft i 2020-årene, og fremover. Åpenhet, oversiktlighet, deltagelse og eierskap er til alle pasienters beste, ikke bare de mest ressurssterke.
