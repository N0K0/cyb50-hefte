\chapter{Ifis tidlige historie}

\author{Skrevet av Narve Trædal}

\section{Røttene}

Lenge var datafaget en aktivitet for spesielt interesserte, fra en sped begynnelse i første halvdel av 50-årene, der flere unge interesserte, blant dem både Ole-Johan Dahl og Kristen Nygaard, som ansatte ved Forsvarets forskningsinsitutt på Kjeller, fikk kjennskap til de nye datamaskinene i Norge. Ved Universitetet i Oslo startet det ved Fysisk institutt, hvor den ``hjemmelagde'' datamaskinen ``Nusse'' ble tatt i bruk i 1953. Maskinen tilhørte egentlig Sentralinstitutt for industriell forskning, SI, som leide rom i kjelleren i Fysikkbygget, men ble også brukt av universitetet.

Opptakten til EDB-senteret (det nåværende USIT) kom litt senere, ved at universitetet kjøpte inn en Wegematic 1000 i 1960.

Som undervisningsfag er det rimelig å betrakte professor Selmers seminar ved Matematisk institutt på midten av 50-tallet som et startpunkt. Ved FFI hold Harald Keilhau kurs i programmering i 1958.

Det var innen disiplinene matematikk og fysikk/ingeniørfag røttene lå. Den tredje komponenten som senere ble konstituerende for fagtilbudet ved instituttet, organisasjons- og administrasjonskunnskap, senere bedre kjent som systemarbeid, var dårlig representert ved UiO.

\section{Redskapsfag eller universitetsdisiplin}

På slutten av 60-tallet og utover i 70-årene var det en nærmest eksplosjonsartet økning i interessen for EDB-utdanning fra studentenes side. Dette var selvsagt et uttrykk for at interessen i samfunnet for dette feltet øket sterkt. I alle offentlige utredninger om teknologisk satsing fra 1965 og framover, står data-området sentralt. Når det gjaldt akademisk, forskningsbasert utdanning, var imidlertid interessen i politiske kretser mindre. Det ble i det alt vesentlige fokusert på kortvarig redskapspreget utdanning. Det førte til at universitetetene (og NTH) langt på veg ble stående alene om å se nødvendigheten av at det nye faget ble gjort til gjenstand for eksperimentell naturvitenskapelig forskning.

Dette var i og for seg naturlig. Den utdanningspolitiske dagsorden i slutten av 60-årene og framover var i det vesentlige preget av utredninger knyttet til etablering av et nytt distriktshøgskolesystem. Ottosen-komiteen la opp til at den framtidige satsingen på postgymnasial utdanning skulle skje i distriktene, ved etablering av to- og tre-årige yrkesrettede utdanninger. Dette ble fulgt opp av regjering og Storting. I DH-konseptet hadde dataundervisning en sentral plass, men vesentlig som redskapsfag innen studieretninger for økonomi og administrasjon. Bare ved Molde, Østfold og Agder DH ble det etablert et 2-årig spesialstudium i EDB. I disse årene foregikk også en kraftig opprustning av den lavere og midlere tekniske utdanningen. Den 2-årige ingeniørhøgskolen ble normen. Mange steder gikk ingeniør-utdanningen inn som en del av distriktshøgskolene. I tråd med Ottosen-komiteens innstillinger ble ressursene kanalisert inn i denne storstilte satsningen på kortere desentralisert utdanning.

I de siste 8-10 år før instituttstiftelsen aksellererte interessen blant studentene for fakultetets datatilbud år for år. Dette skapte store problemer for flere institutter, særlig Matematisk institutt, avdeling D, men også for linjene for kybernetikk og delvis elektronikk ved Fysisk institutt. Tilstrømmingen til mat.nat.-fakultetet forøvrig var i begynnelsen av 70-årene relativt moderat, særlig sammenlignet med resten av universitetet, som også opplevde en
studentboom. Når så et relativt marginalt område ved fakultetet, som data-fagene i realiteten var, fikk en så stor etterspørsel, ble det raskt en sterk ubalanse i undervisnings- og veiledningsbelastningen. De ansatte ved de andre avdelingene ved Matematisk institutt, og storparten av de andre instituttene ved fakultetet, hadde relativt rolige tider, mens deres kolleger ved avdeling D fikk hendene så fulle med utarbeidelse av undervisningsmateriell, undervisning og veiledning, at det ble omlag umulig å få tid til forskningsrelaterte aktiviteter. Særlig gjaldt dette databehandlerne. Og de som i første rekke måtte ri av stormen i begynnelsen av 70-årene var først og fremst professor Ole-Johan Dahl, sammen med universitetslektorene Arne Jonassen og Olav Dahl. Fagretningen for numerisk analyse var ikke fullt så etterspurt.

Ved Fysisk institutt var det lignende forhold. Studentinteressen for kybernetikk var stor. Instituttet befant seg på slutten av 60-tallet plutselig i en situasjon der en stor del av studentene ønsket hovedfag i en fagretning hvor det ikke fantes undervisningstilbud! (For en nærmere beskrivelse av dette henvises til artikkelen om Cybernetisk Selskaps fødsel.) De stillingene som ble opprettet for Lars Walløe, Ellen Hisdal og Rolf Bjerknes, kom som et svar på dette presset. Elektronikk-linjen var også utsatt, men ikke i samme grad som kybernetikk-miljøet.

Dramatikken i denne situasjonen ble for avdeling Ds vedkommende beskrevet i en utredning som ble utarbeidet av alle tilsatte ved avdelingen. Den fikk det malende navnet ``Gjøkungen'', og var et vel dokumentert nødsskrik, hovedsakelig formulert av avdelings-bestyreren, universitetslektor Arne Jonassen. Det er vel ikke urimelig å betrakte den datoen innstillingen ble lagt fram: 9. mars 1974, som unnfangelsesøyeblikket for instituttet, selv om utredningen ikke konkluderte sterkere enn at fakultetet i nær framtid burde vurdere organiseringen av informatikkens administrative plassering på lang sikt. Som man ser, var her informatikk brukt som et samlebegrep for den datarelaterte undervisningen ved fakultetet. I følge utredningen
var det i tråd med hva som var vanlig internasjonalt, særlig i Europa.

``Gjøkungen'' resulterte i at fakultetet satte ned en komité ``for å vurdere datafagenes ressursmessige stilling og administrative plassering ved fakultetet''. Innstillingen fra Informatikk-komitéen, som den ble kalt, kom i juni 1975, og konkluderte enstemmig med at det burde opprettes et nytt institutt bestående av ``numerisk matematikk, databehandling, kybernetikk og digitalteknikk''. Derimot så ikke komiteen noe behov for ``administrativ databehandling'', som komiteen mente var dekket andre steder, bl.a. i Bergen (Handelshøyskolen og Institutt for informasjonsvitenskap).

Informatikk-komiteen ble fulgt opp av utredninger om geografisk samling, og forslag til ny studieplan, og i desember 1975 kunne fakultetet fatte vedtak om instituttstiftelsen med virkning fra 1. januar 1977.

Fra starten av satset det nye instituttet altså på numerisk matematikk, databehandling og kybernetikk, med databehandling og kybernetikk som de særlig populære feltene, sett fra studentsynspunkt. Fra 1.4.1977 ble Kristen Nygaard ansatt som professor II, og rundt ham ble den undervisningen og forskning som senere ble konstituerende for den populære faggruppen for systemarbeid, organisert. I 1980 ble Yngvar Lundh, med hovedstillilng ved FFI, tilsatt som professor II, og det markerte starten på den organiserte undervisningen i digitalteknikk. Tre år senere ble Fritz Albregtsen tilsatt i en NAVF-finansiert laboratorieingeniørstilling. Med det var også Bildebehandlingslaboriatoriet etablert. Dette ble grunnstrukturen ved
instituttet de nærmeste ti årene.

\section{Stillingsressursene}

Ressurssituasjonen var i denne ``svangerskaps''-tiden, såvel som i tiden etter instituttfødselen, fortsatt mager. Informatikk-komiteen hadde konkludert med at et institutt ville ha behov for 29 vitenskapelige stillinger (inklusive 5 II-stillinger) og 3 administrative stillinger. Instituttets behov for teknisk assistanse ble det antatt kunne dekkes av EDB-senteret, samt av 2 rekrutteringsstillinger (vitenskapelige assistenter). Den faktiske situasjonen var imidlertid at miljøene som var aktuelle i instituttet bare disponerte 17 vitenskapelige stillinger (inklusive 2 II-stillinger), 1 kontorstilling og ingen tekniske vit.ass.-stillinger.

Selv om alle syntes sympatisk innstilt til det nye instituttet, så var det altså et stort gap mellom det behovet som ble anslått, og de stillinger som var tilgjengelig. Øremerkede ressurser over statsbudsjettet forekom nesten ikke. Det var stillingsstopp til UiO. De stillinger som ble tilført det nye instituttet, var derfor kun de stillinger som var besatt av de vitenskapelig ansatte som ble flyttet fra Matematisk institutt (avdeling D ble i sin helhet
overflyttet) i tillegg til kybernetikk-gruppen fra Fysisk institutt.

Omlag alle ressurser måtte altså tas ved intern omrokkering av fakultetets eksisterende ressurser - og det er som kjent alltid en tung prosess. Fakultetets dekanus, Tore Olsen, var imidlertid svært innstilt på at prosessen skulle lykkes. Som professor i elektronikk og tidligere bestyrer ved Fysisk institutt hadde han første hånds kjennskap til problemene der, og klarte å få instituttet til å avgi ressurser, sammen med sin fagretning for kybernetikk. Mikroelektronikk-miljøet ved elektronikk-linjen ble beholdt ved Fysisk institutt, selv om det ble understreket at digitalteknikk var et naturlig interessefelt for det nye instituttet. Et særegent problem var de ikke-vitenskaplige stillingene. Et eget institutt forutsatte egen administrasjon og egen teknisk stab. Administrasjonen besto fra starten av en kontorstilling som ble overført sammen med avdeling D, og av instituttsekretær Elisabeth Hurlen som ble nyansatt i halv stilling.

En annen årsak til at det nye instituttet ikke fikk tilført flere stillinger, var at det i årene rundt instituttstiftelsen var tegn som tydet på at studenttilstrømningen ville flate ut. Mange dro derfor raskt den konklusjonen at interessen for data i ungdomsmassen hadde kuliminert. Dette viste seg å være en sterkt forhastet konklusjon. Studentpresset økte raskt til nye høyder. Instituttet styrket stadig sin stilling som det mat.nat.-institutt som hadde det suverént verste tallmessige forholdet mellom lærere og studenter. Selv om instituttet som nevnt møtte en betydelig velvilje i fakultetsledelsen, var det likevel begrenset hva fakultetet kunne bidra med. Likevel øket tallet på ansatte jevnt og sikkert. Ti år etter instituttstiftelsen hadde instituttet kommet opp i 49,5 stilling, dvs. en økning på 30 siden starten. Over halvparten av disse stillingene var blitt tilført via omdisponering på fakultetet. I 1979 hadde vedtatt et ``Program for styrking av fagområdet informatikk'' der man gikk inn for en fordobling av instituttets utdanningskapasitet. Programmets mål-setting, både med hensyn til antall nye stillinger og utdannings-kapasitet, ble oppnådd, men noen bedring i arbeidsforholdene for de ansatte var ikke oppnådd. Fakultetet vedtok et nytt program høsten 1984, ``Program for videre utbygging av fagområdet informatikk'', hvor målsettingen eksplisitt ble satt til en fordobling av antall ansatte ved instituttet. På grunn av knapphetsfaktorer, både hva angikk stillingsressurser og antatt antall kvalifiserte søkere, ble det sagt at det ikke var realistisk å klare mer enn halvparten av denne fordoblingen innen 1990. Det så således ikke lyst ut for en rask forbedring av arbeidsforholdene.

\section{Utstyr}

Tekniske stillinger ble ikke ansett som nødvendig for det nye instituttet. UiOs EDB-senter hadde hele tiden stått for maskinutrustningen, både til studenter og forskere. Ressurssituasjonen var heller ikke slik at det kunne være på tale å bygge opp en egen maskinpark for instituttet. EDB-senteret ytte i 70-årene en betydelig bistand, både teknisk, men også faglig, ved å stå for mye av hovedfagsveiledningen. Da tilstrømningen økte, og det ble opprettet en egen terminalstue for laveregrads studenter i EDB-senterets regi, samtidig som kravene til EDB-senterets virksomhet fra resten av universitetet økte, hendte det at samarbeidsklimaet til tider ble lett anspent. Informatikkmiljøet hadde av og til følelsen av å ikke bli prioritert med sine behov. Det verserer fortsatt historier om at hullkortbunkene til Ifi-ansatte hadde lett for å havne i gulvet på EDB-senteret, dersom man ikke hadde den rette holdningen til de maskinansvarlige. Slike ekstreme hendelser var vel ikke dagligdagse, men det var nok naturlig at interessene til de to datamiljøene skilte lag, etter hvert som kravene fra omverdenen til de to miljøene økte.

Utviklingen av instituttets egen maskinpark og nett skjedde først fra 1980, da Tor Sverre Lande ble ansatt i en amanuensis-stilling. Han hadde i disse årene nærmest eneansvaret for den tekniske kompetansen. Ut over på 80-tallet oppsto det spørsmål om hvilken strategisk utstyrspolitikk instituttet skulle legge seg på. Instituttet samlet seg om en politikk som bygde på distribuerte løsninger med arbeidsstasjoner og servere, basert på programvare som skulle gjøre instituttet i størst mulig grad uavhengig av enkelte maskinleverandører. Mot dette synet sto en annen linje, som langt på veg var den rådende ellers i dataverdenen, nemlig å satse på store sentrale maskiner dominert av en enkelt utstyrsleverandør. EDB-senteret var på denne tiden representant for en slik politikk, som også passet godt inn strategien til f.eks. Norsk Data. Da instituttet i 1982 ble tilkoblet Internett og visst nok som den første i Norge tok i bruk Berkeley UNIX, gikk det således mot strømmen. Utviklingen senere viste at det var en meget framsynt linje, som i dag har fått alminnelig oppslutning, både nasjonalt og internasjonalt.
