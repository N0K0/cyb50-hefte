\chapter{CYB i nyere tid}

\author{Skrevet av Andreas Nyborg Hansen, med hjelp fra Thor Høgås, Odd-Tørres Lunde \& Jan Furulund}

Med nytt bygg, egen studentkjeller og et voksende foreningsmiljø er det mange ønskede og nødvendige endringer som har satt sitt preg på CYB.

Etter at Escape ble åpnet ble det både et ønske og behov for å profesjonalisere foreningen. Med fast ukentlig bardrift, daglig kafé og en svært aktiv arrangementsgruppe ble det tydelig at det måtte være mer fokus på stabilitet og kompetansebygging om man ønsket å opprettholde aktivitetsnivået i fremtiden. Det ble et større fokus på kursing, da ofte kombinert med sosiale arrangementer som hyttetur, middager eller kosetirsdag. Denne perioden fikk også kjellerstyret og hovedstyret faste regelmessige møter, noe som viste seg å være et effektiv tiltak for å stabilisere driften av foreningen. Fruktene av dette arbeidet kan vi se i at Escape ble kåret til årets studentpub av Kronos i 2017\footnote{\url{https://khrono.no/2017/04/escape-vinner-oslos-beste-studentpub}}.

Det er ikke bare CYB som nøt av nye lokaler ved Ifi. Aktiviteten på linjerommene og foreningskontoret førte til at flere og flere foreninger ble stiftet, og foreningsmiljøet som helhet ble mer samlet. Dagen@ifi og CYB har i mange år arbeidet tett, og det har ofte mye overlapp mellom styrene. Året, 2015, med Jan Furulund som leder av CYB, Andreas Lind Johansen som leder av dagen@ifi, og Odd-Tørres Lunde som nestleder i begge foreningene kan en sette startskuddet for en samarbeidskultur som vil leve i mange år fremover. 

Kosetirsdager ble tatt i bruk som en arena for å ta en vennskapelig øl mellom foreninger, og CYB tok et større initiativ for å samle de forening aktive i sosiale lag. Odd-Tørres tok senere over lederrollen i CYB, og tok i bruk samme strategi på store Blindern. En skulle ikke bare bli kjent med de andre kjellerforeningene, de skulle bli venner.  

På Ifi var det ikke bare CYB som ønsket å bygge gode kameratskap. Navet, som på denne tiden hadde et ryktet for å være litt på siden av foreningsmiljøet, jobbet også med å samle foreningsmiljøet med Sigurd Rognhaugen og Henrik Lilleengen i spissen.  Sigurd og Henrik var blant annet sentrale for å innlemme CYB i ``Forente IT foreninger'' (FIF), et samarbeid startet av Abakus\footnote{Linjeforeningen ved Datateknologi og Kommunikasjonsteknologi på NTNU}. Det første møtet til FIF ble holdt i Escape den 4. februar 2018, målet med møtet var blant annet å utveksle erfaringer rundt bedriftskontakt, studiemiljø og pensum ved de ulike læringsstedene. Et godt møte og et nach avbrutt av morgensolen skapte gode bekjentskaper på tvers av landet. 

Høsten 2018 sendte CYB en delegasjon bestående av Thor K. Høgås, Nicolas Harlem Eide, Adrian Helle og Andreas Nyborg Hansen til Online sitt årlige Immatrikuleringsball. Alle fire i deligasjonen ble invitert til Onlines kompilering, dvs. deres opptak av nye medlemmer. I invitasjonen ble de bedt om å ta med klær som de ``ikke er redde for''. Dette burde ha blitt tatt mer seriøst, da en fikk et møte med blant annet kattemat, barberskum og gjørme i løpet av oppholdet i Trondheim. 

Etter en god dusj og kalde forfriskninger dro delegasjonen videre til Online sitt styrevors. I tillegg til de tradisjonelle drikke gavene tok CYB med seg et Bayerfat prydet med CYB sin logo, og signert av en rekke interne. 

De siste år har vært gode for Cybernetisk Selskab. Vi har blitt flere, gjør mer og er mer allemannseie enn vi har vært tidligere. De som driver CYB i dag lever godt av grunnlaget som vår eldre garde har lagt ned, og de som legger ned timer i dag gjør det samme for den neste generasjon. Jeg satser på å høre fruktene av dette i CYB75, og håper CYB100 blir det dobbelte av hva vi er i dag. 
