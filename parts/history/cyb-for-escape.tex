\chapter[Fra dvale til kjeller]{2000-tallet: Fra dvale til kjeller}

\author{Skrevet av Geir Arild Byberg}

I perioden 2006-2007 samlet flere av de nye og eksisterende studentforeningene seg til fellesmøte om fremtiden til Cybernetisk Selskab. Det var et ønske fra ledelsen på Ifi og de som på den tiden satt i styret om å få gjort noe med den labre aktiviteten til foreningen. I verste fall så måtte man få til en verdig avslutning med gravøl for foreningen. Tilstede på dette møtet stilte representanter fra Fagutvalget (FUI), Studentforeningen Mikro, ProsIT, Navet og PING. Etter mye om og men om statusen, veien videre, hvem som skulle styre skuta, den økonomisk situasjonen, etc. ble Geir Arild Byberg valgt som ny styreleder og Heidi-Christin Bernhoft-Jacobsen som ny nestleder. Hver forening på IFI stilte med styreleder\slash nestleder som styremedlemmer.



Dessverre tok ikke aktiviteten seg opp noe særlig de neste månedene. Det var vanskelig å finne fokus på CYB fra hver enkelt, med tanke på andre studentforeninger som ikke kunne eller måtte ofres grunnet den ``nye'' foreningen. Etter en tid ble Geir Arild kontaktet av de tidligere styremedlemmene Ole Kristian Hustad og Martin Lilleeng Sætra som lurte på om det skjedde noe aktivitet i CYB og om skuta var virkelig på vei ned. Med litt motivasjon begynte det å vokse opp et lite håp og tro på at CYB faktisk kunne bli noe. Det ble gjennomført en ny (og skikkelig) generalforsamling hvor det nye styret bestod av Geir Arild Byberg (styreleder), Øyvind Bakkeli (nestleder), Eirik Hjelle (webmaster), Ole Kristian Hustad (gammel pamp 1), Anna Dahl (gammel pamp 2) og Martin Lilleeng Sætra (gammel pamp 3). Mens Geir Arild, Øyvind og Eirik bygget opp en ny webside til CYB arrangerte den eldre garde CYB sin 35-årsdag i kjelleren på Café Abel. Samme dag var det by:Larm i Oslo og kvelden fortsatte i byen.

Nå hadde CYB fått i gang et styre og nå begynte fokuset på å dra i gang aktiviteter. ``Drit i hvor mange som kommer, la oss heller gjennomføre det!'' ble mottoet. Websiden ble et godt brukt forum, vi fikk vekket interessen for foreningen blant studentene og interessen for CYB begynte å ta seg til. Dette var nok i en fin kombinasjon av at for første gang på lenge hadde endelig Ifi en studentforening som var sosialt rettet og som var for alle studentene, uavhengig om du var bachelor, master, profesjon eller enkeltemnestudent!

Ikke lenge etter at CYB begynte å våkne til liv ble CYB forespurt av styret i Realistforeningen om vi var interessert i å arrangere noe informatikk-vennlig i April måned i det herrens år 2009. April hadde blitt valgt ut til å være Informatikk-måneden med faglige foredrag (dette var året da The Pirate Bay måtte bevise sin uskyld) og sosial moro. CYB stilte opp med å arrangere et faglig seminar med navn That’s IT! Problemet med arrangementet var at det var planlagt avholdt torsdag-ish 30. april og med tanke på at 1. mai var rett rundt hjørnet ble oppmøtet så som så. Men for dem som møtte opp ble det kake, pizza (omtrent like mange pizzaer som på dagen@ifi) og øl! Og alt i alt fikk alle en hyggelig opplevelse og ryktene har seg enda til at mange skulle ønske de kunne være der…

I løpet av året vokste CYB videre og arrangerte tidenes første Ping Pong turnering, tidenes første whiskysmaking, julelodding, juleprogrammering, bidro sterkt til fadderukene sammen med de andre studentforeningene og mye, mye annet. Men i det skjulte pågikk det også en annen liten hemmelighet: informatikk-studentenes første kjeller i det nye bygget!

Når diskusjonen startet anes ikke men ledelsen i Ifi, med Morten Dæhlen, Line Valbø, Narve Trædal og Terje Knudsen i tet, sørget for at det nye bygget skulle ha studentkjeller. Noen av de andre studentforeningene ble spurt men grunnet lite mulighet til å ta opp ballen og følge opp ble det enighet om at CYB skulle få ansvaret for å drive kjelleren. CYB måtte bare ha tid til å ``våkne til live''. Som del av denne vekkelsen hadde Magnus Johansen startet på Ifi, blitt CYB sin første Kjellermogul og sørget, sammen med sin trofaste gjeng i Kjellerstyret, for at avtaler ble inngått, personell (les: frivillige studenter) rekruttert og ivaretatt og drømmen om en egen studentkjeller nærmet seg nå en realitet!
