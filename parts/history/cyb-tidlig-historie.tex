\chapter[CYB tidlig historie]{CYB frem til midten av 2000-tallet}

\textit{Red.anm.: Denne teksten er hovedsakelig basert på bidrag man har fått fra alumni, tekster i CYBs 25-års jubileumshefte og det man ellers har funnet i gamle arkiv. Det er noen hull i historien her og der, mye har rett og slett blitt glemt, og med til tider glissen dokumentasjon er det ikke alt som kommer klart frem. Men håpet er at det man har funnet gir et godt inntrykk av hvordan de aktive opplevde CYB sine første 35 år.}

\author{Skrevet av Arne Hassel}

Cybernetisk Selskab har hatt en jevn utvikling av hva foreningen konkret har jobbet med, men fokuset har alltid vært på studentene, enten det var dem tilknyttet linjen Kybernetikk eller alle tilknyttet Institutt for informatikk. Det sosiale har stått i sentrum, men frem til arbeidet med Escape var det også sentralt å skape gode faglige rammer, noe som også inkluderte kontakt med næringslivet og andre aktører som kunne tilby arbeid etter studiene. 

Tilknytningen til institutt for informatikk skjedde tidlig i instituttets historie. Kollegiet ved Universitetet i Oslo vedtok 1. oktober 1976 opprettelsen av Institutt for informatikk, med virkning fra 1. januar 1977. Studentene var ikke sene med å reagere og allerede 11. november 1976 ble det vedtatt på Generalforsamlingen (med 18 mot 7 stemmer) at CYB skulle ha tilhørighet til Institutt for informatikk istedenfor Fysisk institutt.

Et annet vedtak som skjedde raskt etter opprettelsen av Ifi ble bestemt var beslutningene som ble gjort 25. oktober 1976. CYB og Fagkritisk Gruppe ved databehandling ved matematisk institutt (FKG) hadde et felles styremøte hvor man bestemte at man skulle skaffe representanter til det kommende instituttstyret. Det ble også bestemt at man skulle opprette et fagutvalg på informatikk. Møtet betegnes som historisk i referatene. 

Rollen til fagutvalget skulle være å opprette kontaktpersonordning, se på undervisningssituasjonen og ressursbehov og administrere lesesalsplasser på informatikk. For både instituttstyret og fagutvalget skulle CYB skaffe to representanter, mens FKG skulle skaffe tre representanter. Funksjonene som CYB bidro til her har i senere tid blitt ivaretatt av Fagutvalget ved Ifi.

Selv om man tidlig hadde tilknyttet til instituttet tok det hele 33 år før man valgte å kalle seg instituttforening. Hvordan man har valgt å omtale seg har endret seg litt gjennom tidene, men det har gått i variasjoner som ``studentforeningen ved Institutt for Informatikk'', ``informatikkstudentenes forening'' og ``en faglig forening for alle informatikkstudenter''. Men den 27. november 2009 vedtok CYBs generalforsamling å kalle organisasjonen instituttforening.

\section{Debatt og foredrag}

De aller første årene hadde man stor suksess med debatter myntet mot allmennheten. Spesielt debatten ``Kan datamaskinen erstatte politikeren?'' som ble arrangert 5. oktober 1969 var viktig, hvor CYB for anledningen hadde trykket opp 3000 løpesedler. 250 mennesker møtte opp på Frederikke i håp om å få svar på spørsmålet om datamaskinen kunne forutsi de samfunnsmessige utslagene av politiske avgjørelser. 

Panelet besto av to politikere\footnote{Statsråd Helge Seip og stortingsmann Toralf Westermoen}, to samfunnsvitere\footnote{Forskningsleder Finn Solie og dr. philos. Jens A. R. Christophersen} og to kybernetikere/databehandlere\footnote{Amanuensis Lars Walløe og professor Ole-Johan Dahl}. Temaet hadde blitt foreslått av Jens Balchen fra NTH (tidligere navn på NTNU), som også holdt innledningen til debatten via telefon fra Trondheim, da han ikke kunne ta flyet pga tåke. Seansen ble ledet av Per Øyvind Heradstveit, programsekretær i NRK. Debatten belyste et tema som godt kan diskuteres den dag i dag, og et godt poeng fremført av Ole-Johan da som også er viktig i dag, er at datamaskinens viktigste oppgave er å gjøre informasjon mer tilgjengelig.

En viss suksess hadde man også to år senere da man arrangerte et møte 13. oktober 1971 med temaet ``Medisinsk databehandling''. Møtet fikk bra dekning i Aftenposten. Men utover dette ser det ikke ut til at man fikk til den store mediedekningen rundt debatter og møter. Dette til tross for at man gjennomførte en konferanse 9. april 1973 med temaet ``Kybernetisk Krigføring'', hvor professor Johan Galtung og forskningssjef Erik Klippenberg deltok. Med Vietnamkrigen som bakteppe ble diskusjonen opphetet, faktisk så het at debatten ble ufin, medlemmer fra Polyteknisk Forening forlot møtet, og man besluttet å avslutte konferansen.

\section{Ekskursjoner}

Om ikke debatter og konferanser var det CYB skulle bygge tradisjoner rundt, så var turer og ekskursjoner en viktig traadisjon for CYB i mange år. Spesielt viktig var turen til Servomøtet, som startet allerede i 1969 og holdt på annethvert år frem til og med 1995. Servomøtet var (og er fortsatt) en faglig konferanse med fokus på kybernetikk og reguleringsteknikk, og var nok veldig relevant for CYB i sine tidligere år, men kanskje ikke så mye mot slutten av tradisjonens levetid. Uansett sammenfalt konferansen med UKA i Trondheim, noe som var en gyllen anledning til å kose seg med både fest og revy i tillegg til det faglige.

En av grunnene til at Servomøtet var en tradisjon for CYB så lenge var nok også at Rolf Bjerknes var en ivrig pådriver. Rolf var en kjær lærer på Ifi og ble av flere kalt ``Onkel Rolf''. Men han var også en sentral skikkelse i CYB, som ivrig tilhenger (og iblant pådriver) av foreningens arbeid og fast inventar på Generalforsamling hvor han holdt populærforedrag. Han var også gjerne med på flere av turene til Trondheim og ofte med på bedriftsbesøk som CYB arrangerte. Han hadde en egen evne til å folkeliggjøre kybernetikk, noe som viste seg i hans til tider spesielle fokusområder som gjorde at enkelte spøkefullt kalte han ``Mannen som fant komplekse egenverdier i kloakkrenseanlegget på Jessheim''. CYB valgte da også å sende representanter i Rolf sin begravelse i 2017.

CYB var som nevnt ivrige arrangører av bedriftsbesøk og ekskursjoner. Man hadde til tider månedlige besøk på instituttet hvor bedriftsrepresentantene holdt foredrag, men dro også ofte ut på besøk til bedriftene. Eksempler på dette var flere turer til NTH (også utenom Servomøtet), Christian Michelsens institutt i Bergen, Kongsberg Våpenfabrikk\footnote{Senere avviklet men innlemmet i Kongsberg Gruppen og Kongsberg Defense \& Aerospace}, Norcontrol på Horten\footnote{Også innlemmet i Kongsberg Gruppen} og Borregaard i Sarpsborg\footnote{Produksjon av flere kjemiske produkter, i dag et bioraffineri}. På et tidspunkt hadde da også CYB en egen buss som man brukte til å dra på ekskursjonene.

\subsection{Turer til utlandet}

Det var også populært å dra til utlandet, noe man gjerne gjorde i årene mellom Servomøtet. Det var spesielt på 80- og 90-tallet at man tok kontakt med universiteter og bedrifter i utlandet for evt besøk.

I 1984 gikk den første turen vi vet om til USA, hvor ni deltakere besøkte Silicon Valley for flere besøk, blant annet til DEC\footnote{Digital Equipment Corporation, som i 1998 ble innlemmet i Compaq} noe som var spesielt interessant da Ifi på det tidspunktet hadde DEC 10-maskiner. Man fikk også tatt turen innom National Computer Conference som ble arrangert i Las Vegas. Og når man var i USA måtte man jo også utforske andre ukjente kulturfenomen, som f.eks. McDonalds som på det tidspunktet enda ikke fantes i Norge\footnote{Ble stiftet i 1988}.

USA-turen ga mersmak, og i 1988 ville man gjenta suksessen. Men for denne turen har Morten Moen og Ole Christian Lingjærde skrevet en lengre tekst, så les deres kapittel om du ønsker mer innsikt.

Danmark var neste destinasjon etter de første USA-turene. Den første turen gikk til universitetet i Aalborg, med program for konferanse fra onsdag til fredag, mer sosialt program for lørdagen, og så hjemreise på søndagen. Den andre turen gikk til København og var nok mer fokusert på det sosiale, med planer om å besøke Tuborg bryggerier, John Ripleys Bizarre Museum, Eksperimentariet og Planetariet og ikke minst et erotisk museum\footnote{Det er dog bare bryggeriene man vet med sikkerhet ble besøkt, de andre stedene bemerkes ikke i hva folk husker fra den tiden}.

I 1997 gikk turen til Helsinki for å besøke universitet og Nokia. Hele 24 deltakere fikk da muligheten til å være med på forelesninger og nettverke med både studentforeningen ved informatikk (som hadde egen pub) og representanter fra Nokia (som tok dem med til VIP området dems), være det gjennom studentballet Tietokilta eller sauna, som både studentene og Nokia selvfølgelig var behjelpelige med.

En siste tur fikk man til i 1999, da man nok en gang vendte nesen mot USA. Ikke altfor mye vites om denne turen, men den gikk til San Francisco, hvor man blant annet fikk besøke en serverhall. Rommet de fikk se var kanskje 100x100 meter, med digre UPS batterier og avlåsbare skap hvor firmaer hadde rack. Her fikk man vite at blant andre Altavista hadde sine servere.

Utover disse turene vites det ikke hva man fikk til, men det var mange tanker og drømmer om hvor man ville dra. Et av ønskene var å dra til Sovjet, men det ble bare med ønsket.

\section{Bedriftskontakt og bedriftsmedlemmer}

En sentral del av Cybernetisk Selskab sin virksomhet gjennom historien var fokuset på det faglige gjennom organisering av foredrag og knytte kontakt med næringsmiljøet. Sistnevnte skjedde primært gjennom bedriftsmedlemskap og noe sponsing. Sponsing var litt mer fleksibelt før, f.eks. Oslo-bryggeriene som sponset øl til generalforsamling.

Hvor strukturert arbeidet mot bedrifter er vanskelig å vite sikkert pga manglende dokumentasjon, men arbeidet ble vedtektsfestet i 1971, og arkiver tilbake til 1986 viser at man på det tidspunktet hadde rundt 15 bedrifter. Videre hadde man et par topper i 1988 og 1990 med hhv 44 og 41 bedrifter, og ellers ser det ut til at man lå mellom 15 og 30 bedrifter i semesteret. Bedriftsmedlemskap holdt man på med helt til 2002, da dot com-boblen hadde gjort sitt til å svekke mange bedrifters økonomi. 2002 var også året hvor man sluttet med vervet bedriftsansvarlig. Man tok riktignok opp vervet igjen i 2007, men det var nok mer med tanke på det arbeidet man bidro med for å få i gang Navet.

På det meste gjennomførte man egne bedriftsdager en gang i semesteret. Her fikk bedrifter komme på besøk og presentere seg for studentene. Det kan se ut som omfanget var rundt fem-seks bedrifter, så nivået var ikke på høyde med nåtidens dagen@ifi. Men konseptet var mye det samme, nemlig at studentene fikk en mulighet til å bli kjent med bedriftene gjennom presentasjoner og en sosial ramme rundt med mat og drikke.

Pengene man fikk inn gjennom bedriftsmedlemskap speilet på mange måter hvordan det gikk i bransjen generelt. Gjorde bedriftene det bra brukte de gjerne penger på å profilere seg og å komme i kontakt med studentene. Et bedriftsmedlemskap var heller ikke spesielt dyrt, 350 kr på 80-tallet og 2000 kr på 2000-tallet, men det var såpass inntektsbringende at man klarte å spare seg en del penger. Mye av dette ble satt på det som ble kalt CYB-fondet, som skulle brukes når CYB trengte det til større initiativ.

\section{Internt foreningsliv og sosiale arrangement}

Det var ikke bare det faglige som sto i sentrum for de aktive i CYB. Det sosiale har alltid vært en viktig bestanddel av CYB, noe som vitnes om allerede på 70-tallet da det var det blitt tradisjon for sporadiske Selskabs-samlinger i Blinderkjeller’n i Fysikkbygget og gjerne noen rundturer på byens bryggerier som Frydenlund, Schous og Ringnes for å sikre rimelig (helst gratis) pils til tørre student-struper. Frem til man flyttet til det nye Ifi-bygget i 1988 (som nå er omdøpt til Kristen Nygaards hus) hadde CYB kontor i brakkene utenfor Fysisk institutt, og i tillegg til å fungere som leseplass og kontor hadde det også en viktig funksjon som lagerplass for øl og festlokale.

Da instituttet flyttet inn i det nye Ifi-bygget i 1988 satte det mye av rammene for internlivet for de engasjerte informatikk-studentene. Man fikk kontorer i første etasje, et lite rom på rundt 20 kvm som hovedsakelig ble brukt som lagerplass av de forskjellige foreningene som delte rommet. Det var ikke før man fikk kjøpt en sofa på Frelsesarmeen i 1998, ti år etter at man hadde flyttet inn i bygget, at man fikk det skikkelig sosialt på kontoret og det ble en plass for de interne å ha møter og planlegge\footnote{Og alt annet man kan finne på å gjøre i en sofa som står gjemt i et lite rom\dots}.

Et produkt av denne møtevirksomheten var Foajéfesten, som startet i 1998 og gjennomført noen semestre utover 2000-tallet. Festen spente seg over tre etasjer, med servering og dansegulv og flere aktivitetsrom. CYB fikk etterhvert ganske god rutine for gjennomføringen av dette, noe som nok også kom dagen@ifi til gode da man startet det i 2003.

Det sosiale skapte også gode rammer for flere tradisjoner, deriblant Rekeaften som ble arrangert hvert vårsemester (med til tider kreative løsninger, som f.eks. å tine reker i dusjen på tilfluktsrommet) og noen høstsemestre. Når det startet vet man ikke helt, men man finner dokumentasjon helt tilbake til 1989 og til og med våren 1998.

Ekstraordinær generalforsamling var på et tidspunkt ikke spesielt ekstraordinært, siden det i mange år ble arrangert fast i starten av hvert semester. Det var også egne tradisjoner for mat tilknyttet ordinær og ekstraordinær generalforsamling (noe man holder i hevd enda); på førstnevnte var det eggerøre, loff og spekemat som gjaldt, med øl og akevitt til drikke, mens det på sistnevnte var mer spartansk med smørbrød og te.

Det var selvfølgelig nachspiel etter de tradisjonelle møtene og historier fra disse viser at det var godt liv. Eksempler på dette er den gangen man fikk brannvesenet på besøk pga en serviett som hadde havnet oppå et telys, noe som uansett ikke satte en stopper for stemningen siden flere studenter senere badet i nettoen i fontenen ved siden av den gamle kantina.

Visning av filmer var også populært og på et tidspunkt var filmkveldene i Store Auditorium blant de mest populære arrangementene, med surround lyd og import film på LaserDisc. Der viste man klassikere som Fight Club, Indiana Jones and the Last Crusade, The Blair Witch Project, Get Shorty og Twister. Litt trøbbel ble det da man viste Waterworld før den hadde premiere på kino, men fulle saler ble det. Et foredrag man fikk til rundt dette var med en av skaperne til barne-TV produksjonen ``Pompel og Pilt'', Arne Mykle, med påfølgende visning av hele serien på storskjerm slik at man kunne gjenoppleve traumatiske minner fra barndommen.

En annen populær bruk av Store Auditorium og storskjermen var gaming-konkurranser. Her ble det satt opp turnering i Tekken og andre populære spill, og gjerne i forbindelse med storfestene man arrangerte.

Man feiret også seg selv og instituttet opp gjennom årene. Som del av 10-års feiringen av Ifi i 1987 hadde man foredrag, fest og minirevy på terminalstua i Fysikkbygget. På selve 25-års dagen til CYB, 17. februar 1994, feiret man med marsipankake og champagne, etterfulgt av fire episoder av Pompel og Pilt på storskjerm. Man hadde også en jubileumsuke, hvor man gjennomførte IN-festen, foredrag og flere arrangement, og ble avsluttet med en storslått jubileumsmiddag på Teknisk Museum. Med trykking og salg av jubileums-t-skjorter som prydet en fargelagt versjon av CYB-logoen, ble 1994 også et rekordår med antall medlemmer - hele 604 personer (en rekord som holdt seg frem til Escape åpnet).

Sist, men ikke minst, så har også turer til studenthyttene i Nordmarka vært en tradisjon i CYB. Hvor langt tilbake denne tradisjonen går vites ikke, men man finner referanser til hyttetur tilbake til 1997, da med tittelen ``Nå spirer pilsen!'' Turene til Nordmarka vet man med sikkerhet startet i 2005, da man begynte å dokumenterte turene med kamera\footnote{Bilder av som regel veldig festlig art, men som ikke er nødvendig å dele i denne boka}. Denne tradisjonen har holdt seg i hevd, og CYB prøver i dag å arrangere minst en hyttetur i semesteret\footnote{Mange av disse turene gikk til KSI-hytta, som dessverre brant ned høsten 2018}.

\section{Drømmen om en egen Ifi-kjeller}

Selv om man ikke hadde egen studentkjeller sto det ikke på viljen med å arrangere fester. Et eksempel på dette er den nevnte IN-festen, den tradisjonelle festen for informatikerne som ble arrangert i samarbeid med Realistforeningen på Vilhelm Bjerknes, hvor man blant annet hadde Cyberlympics i diskettkasting (5 1/4'' disketter), ``IFI-GOGO BAR'' og mye mer. 

Man fikk også til en pub i kantina på Abel i 2002, som man satte opp ifm den tradisjonelle pub-til-pub runden. Puben var en kjempesuksess og satte nok flere griller i hodet på folk. Man prøvde også å stille med P2P-bar på Ifi, hvor man da fikk benytte seg av skjenkebevillingen til SIO-kantinen. Selv om det etterhvert ble gode rutiner for både det og gjennomføring ellers, så var det tydelig at bygget ikke var særlig egnet for fest. Bomberommet ble også tatt i bruk for pub, men krevde mye arbeid å sette opp og ikke minst rydde opp etterpå. Men det sto ikke på forsøkene, en gang satte man også opp fakler over gangbrua fra campus for å lokke pubgjester.

Det var mange drømmer om egen studentkjeller, noe som også beskrives i jubileumsheftet fra 25-års feiringen. Dette resulterte at CYB-fondet etterhvert fikk en veldig bestemt oppgave, nemlig å fungere som ekstra egenkapital til den fremtidige kjelleren. Fondet hadde bygget seg opp til litt over 100 000 kr, og man gjennomførte fortsatt nyvalg på fondsbestyrerne med jevne mellomrom. Ikke at dette i praksis ble gjennomført - da man skulle få ut pengene ifm oppstarten av Escape så viste det seg at man hadde glemt å oppdatere fondsbestyrerne i banken, så det ble litt arbeid med å få ut pengene igjen. Fondet var nemlig egentlig en vanlig bankkonto med navnet Cybernetisk Selskab Fond. Da CYB fikk foretaksnummer i 2006 så glemte man å melde i fra til Nordea, og da man ville ha ut pengene i 2009 ble det krøll. Etter mye frem og tilbake med bank og tidligere fondsinnehavere klarte man til slutt å overbevise Nordea om at kontoen med navn Cybernetisk Selskab Fond faktisk tilhørte studentforeningen Cybernetisk Selskab.

Aktiviteten og tradisjonene som CYB hadde bak seg var avgjørende for at foreningen fikk æren av å ta ansvar for driften da det ble klart at studentkjelleren endelig skulle bli realitet. Dette til tross for at foreningen på midten av 2000-tallet hadde begynt å slite med lite aktivitet og færre og færre medlemmer. Det at man i over 30 år hadde vist at man var til for alle studentene på Ifi var avgjørende, og det fokuset man hadde på å skape gode sosiale rammer for studentmiljøet var (og er) en hjørnestein i foreningen.