\documentclass[../main.tex]{subfiles}

\begin{document}
I den neste delen komme litt av den nyere Cyb historien, hvor gikk veien videre etter 1994?

\subsection{Veien videre (95-99)}
av Knut Erik Borgen

Om jeg husker riktig så begynte jeg i styret i Cybernetisk Selskab i 1996. Leder var da Stein Otto Grimstad og han så at det kunne være interesse for å vise film i Store Auditorium. Auditoriumet var utstyrt med projektor og vi hadde via kontakter tilgang på laserdiskspiller som ga bildekvalitet som man bare kunne drømme om utenfor en kinosal. Lyden var heller ikke noe å si på, fordi hi-fi-entusiaster i styret dro med seg digre høytalere med surround-forsterkere som bidro til lydopplevelsen.

Filmkveldene ble raskt veldig populære og store auditorium ble fylt til randen. Med import fra USA ble det til og med norgespremiere på Waterworld og salen ble fullsatt med kun reklamering på IN105 forelesningen. Andre populære kvelder var visning av alle Star Wars filmer, Aliens og andre klassikere, og det ble som regel en til to filmkvelder i perioden. Filmkveldene gjorde altså at medlemsmassen i Cybernetisk Selskab igjen ble mer en 40 medlemmer…

I vårsemesteret i 1998 inntraff det en viktig hendelse i Cybernetisk Selskab. Cybernetisk Selskab hadde et lite rom utenfor allfarsvei nær bomberommet, som for det meste ble brukt som lager. Dette endret seg brått da det ble innkjøpt hjørnesofa fra Frelsesarmeen. Hjørnesofaen passet som hånd i hanske i rommet, dvs på millimeteren, og nå ble det sosialt å være i styret. Vi fikk et møtepunkt hvor vi møttes i pauser og sofaen ble straks byttet ut mot lesesal.  Kaffeglasset i kjøleskapet begynte nå å gå fullt titt og stadig, da det ble enkelt å ta seg en pils ut av kjøleskapet, og vi fikk egen kaffeansvarlig som kjøpte kaffe.

I vårsemesteret i 1998 hadde vi et knasende bra styre og aktivitetsnivået var høyt. Da bestemte vi oss for å arrangere foajéfest. Sigurd Mytting og undertegnede tapetserte Blinderen med flyers og reklame og stemningen steg før den store dagen. Dette skulle bli en sikkelig fest:
Lyd og lysutstyr ble leid inn til dansegulv i foajeen på IFI, tappetårn fra Ringnes; med15 fat øl, egen hooch/vin/rusbrus bar drevet av Verdande. pølse og snacks bod ved inngangs døra og sist og ikke minst Nerd zone i Store Aud (mame på storskjerm med joypad).

Festen ble en braksuksess med rundt 300 besøkende og en time før sjenkeslutt var vi gått tom for fatøl.

I tillegg til de store festene ble det også arrangert koselig /local/pub/ i bomberommet i samarbeid med Verdande. Her ble bomberommet omgjort fra trimrom til pub ved enkelt å flytte ut alt av trimutstyr og dra ned stol og bord fra 3 etasje og inn på bomberommet, Etter festen, typisk dagen etter, så ryddet man trimrommet tilbake. Litt rundt regnet så ble det lagt ned 24 timer arbeidsinnsats før og etter for å få til dette. Alt av øl ble kjøpt inn via bil fra Smart Club og vin fra polet (det går faktisk an å frakte  24 kasser pils i en vanlig Mazda 929). Omsetningen var ikke alltid så mye å skryte av, men alle var enige om at det i det minste var en hyggelig internfest for Cybernetisk Selskab, Verdande og Ping.

Bedriftsdager ble det også arrangert. Frem til IT boblen sprakk i år 2000 satt pengene løst i bedriftene og bedriftene betalte gladelig for å komme å presentere seg. Dette gjorde at fondskontoen til klubben økte jevnt og trutt.

I perioden 1996 til 1999 ble det også arrangert to utenlandsturer i regi av Cybernetisk Selskab. Den ene gikk til Helsinki i Finland og den andre til San Fransisco i USA.

\subsection{Kampen om IFI2}
av Iver Stubdal

Helt siden informatikkbygget sto ferdig på 1980-tallet hadde det vært klart at det var for lite. Det manglet kontorplasser, og enda verre, studentene var henvist til at gruppeundervisningen foregikk i midlertidige brakker og tilfeldige seminar rom rundt om på Mat-Nat. Som mange andre drømte også Cyb om den dagen da alle IFI-studenter kunne samles i ett bygg, og i én studentkjeller.

I 1998 hadde Stortinget lovet at det skulle bevilges penger til å bygge et Ifi2, men 3 år senere var det fortsatt ikke lagt så mye som en krone til prosjektet på bordet. Da det heller ikke på budsjettet på 2001 var plass til noen post for ifi2 var begeret fult for informatikerne. På et hastig allmøte på IFI ble det satt ned en impromptu aksjonskomité som skulle organisere en protest foran stortinget, bestående av 2 vitenskapelige ansatte og 3 fra Cyb.

Komiteen gikk raskt i gang med sitt virke, og på Cyb kontoret ble det skrevet slagord og paroler. Inne på bomberommet på IFI, hvor Cyb mange ganger hadde avholdt improvisert IFI-pub, ble det nå laget protestbannere og plakater som til slutt skulle bidra til å skaffe foreningen en ekte studentkjeller. Komiteen fikk også den geniale ide å sette opp et innleid hoppeslott foran Stortinget for å illustrere det luftslottet ifi2 var blitt. Med postere og innlegg på forelesninger ble ifi studentene mobilisert.

Så en regntung dag i slutten av november samlet rundt 100 engasjerte IFI studenter og ansatte seg foran et luftslott på Løvebakken og forlangte svar fra politikerne. Taktfaste rop og stormende appeller kalte representanter for Stortinget ut på plassen.

%TODO bilde, banner utenfor Stortinget

Ingenting ble lovet der og da, men snart kunne resulter av demonstrasjonen sees da det på revidert statsbudsjett ble bevilget penger til forprosjektering av et nytt informatikkbygg. Det vedtaket ble fulgt opp med videre bevilgninger til oppføring av bygget, først dekket bevilgningene knapt denne Lego modellen, og til slutt etter mange trange år kan både ifi og Cyb endlig flytte inn i eget hus.

%TODO bilde, Lego-modell av IFI2
%TODO bilde, IFI2 (ikke i Lego)

\subsection{Fornebu-paradokset}
av Ole Kristian Hustad

Det ble på ett tidspunkt (98-99) forslått å flytte Informatikk faget ut til den nye “it-klyng” på Fornebu, på skrivebordet så det sikker ut som en fin plan, men når det ville medført at “alle” Informatikk studenter daglig måtte pendle de ti kilometerne ut til Fornebu mellom forelesninger (det høres kanskje ikke mye ut, men av erfaring vet undertegnede at den reisen fort tar 50 min når gangavstand og kø blir inkludert) virket det ikke så forlokkende allikevel, heldigvis ble det forslaget forkastet.

I den senere tid (2011)  har også en rekke næringslivstopper uttalt seg negativt om fornebu prosjektet da det har vist seg at det ikke er nok å bare planlegge “it-klynger” og at de på nesten magisk vis skal skal føre til innovasjon, noen må faktisk gjøre noe også, og den virkelige utfordringen er at det er ikke alltid så lett å se hvem som kommer til å legge det neste gullegget.

\subsection{dagen@ifi - litt historie}
av Anna Dahl

Det hele startet faktisk med Verdande (foreningen for kvinnelige IFI-studenter). I 2003 var foreningsstatusen på IFI ganske begredelig, med lav aktivitet i de fleste foreningene. Verdande var i en særlig dårlig stilling ettersom alle i styret var ganske enige om at grunnlaget for å ha en egen "jenteforening" på IFI var et helt annet enn i 1997, da foreningen ble startet. Samtidig var vi klare over at også de andre foreningene slet med liten oppslutning, få (styre-)medlemmer og lavt aktivitetsnivå.

22. september 2003 sendte derfor undertegnede dinosaur, daværende leder i Verdande, en epost til Cyb, Ping og FUI. Utdrag følger:

«Det kan se ut som om viljen til å engasjere seg er synkende blant IFI-studenter. Ut fra det vi har hørt (og ser på de forskjellige websidene), sliter de fleste av oss med rekrutteringen, og det blir holdt atskillig færre arrangementer enn for bare få år siden.

Vi kan sikkert skylde på økt arbeidspress blant studenter, kvalitetsreform og det ene med det andre, men det er nok også på tide å se litt nærmere på hvilket tilbud vi samlet sett tilbyr studentene. Studentforeninger er til for studentene -- vårt tilbud skal være med på å gjøre det mer sosialt, bedre, morsommere og lettere å studere ved Ifi. Kan vi egentlig si at vi oppfyller våre egne målsetninger, slik situasjonen er i dag?

Verdande har lenge sunget på siste verset, og vi som sitter i styret i dag, tror ikke at det er "marked" for en egen forening for jenter ved Ifi. Derimot tror vi at det burde være fullt mulig å skape et fantastisk miljø for studentengasjement ved Ifi, i samarbeid med dere! Tanker om sammenslåing av studentforeningene på Ifi har såvidt vært luftet tidligere -- nå er det kanskje på tide å snakke skikkelig om det?»

Eposten fikk umiddelbart positiv respons fra leder i Cyb, Eirik Munthe, og både Fui og Ping sa seg villige til å delta. Etter hvert ble mailen videresendt i alle retninger, og både leder i ProsIT (Knut Johannes Dahle) og Mikro (Omid Mirmotahari) meldte sin interesse.

29. september ble det første «foreningsmøtet» holdt, i styrerommet på IFI1. Det ble en lang diskusjon - mange var positive til å slå sammen foreningene, men det kom også en del motforestillinger. Situasjonen for de nyere («nisje»-)foreningene var en litt annen enn for de eldre: de slet av ulike årsaker ikke like mye med rekrutteringen, hadde godt aktivitetsnivå og så dermed ikke det samme behovet. I tillegg kom det opp en del praktiske utfordringer som endringer av vedtekter, valg av navn, diverse generalforsamlinger og nyvalg etc.

Etter hvert ble det klart at full sammenslåing ville koste mer enn det ville smake, men samtidig var det åpenbart at alle ønsket mer samarbeid og samkjøring av aktiviteter og arrangementer. Så dukket idéen opp: hva med et samarbeidsprosjekt i stor skala, der alle foreningene ville få profilert seg overfor studentene og instituttet, og samtidig testet ut hvordan samarbeid kunne fungere? Dette var det stor stemning for - og langt enklere (vel..) å implementere, enn sammenslåing.

En styringskomité bestående av 12 personer ble satt: Dag-Erling Smørgrav, Dagfinn Ilmari Mannsåker, Eirik Munthe, Håvard Moen, Hege L. Pedersen, Kaja Elisabeth Mosserud, Mads Andre Bergdal, Omid Mirmotahari, Per Andreas Norseng, Peter J. Korsmo, Tor Sigurd Mytting og undertegnede. Prosjektet ble først kalt «Den Store Ifi-Dagen (Og Natten) - DSID(ON)», i god informatiker-ånd.

På det første møtet i styringskomitéen foreslo Hege «Dagen@Ifi» (senere dagen@ifi) som navn på arrangementet, og man fant en dato: torsdag 30. oktober. Programmet ble planlagt: faglige foredrag på dagtid, kombinert med stands i fellesarealene fra 12-16. Foreningene skulle ha felles stand, og alle forskningsgruppene skulle inviteres til å sette opp egne. I tillegg ble man enige om å ta kontakt med diverse bedrifter og høre om de kunne være interesserte i å sponse arrangementet, mot å få sette opp egne stands.

Tanken var å holde studentene på Ifi hele dagen, så det var åpenbart at man trengte et trekkplaster som bindeledd mellom aktivitetene på dagtid og kveldstid i tillegg til servering, og mange forslag til gode populærvitenskapelige foredragsholdere kom opp. Eirik Munthe foreslo å hyre inn en stand-up-komiker, og fikk i oppdrag å finne en slik. På kveldstid skulle det så bli fest: DJ/dans og ølservering i foajéen og diverse aktiviteter i auditoriene og grupperommene/korridorene - spill, karaoke, konkurranser og ikke minst den Entrapment-inspirerte laserkorridoren satt opp av Omid Mirmotahari og Mikro.

Sjelden har det vel blitt «blestet» mer aktivt for et arrangement, enn for det første Dagen@Ifi-arrangementet. Siden det var helt nytt, var det høyst usikkert om folk ville synes konseptet var interessant nok til å dukke opp. Helt fra starten på arbeidet i september ble det hengt opp mengder av plakater overalt på Ifi og resten av realfag-byggene på Blindern, man gikk innom minst én forelesning i hvert eneste fag som hadde forelesninger i perioden september-oktober, mail ble sendt på alle lister, og for første gang ble det gitt tillatelse til å bruke plass på utskrifts-forsidene til noe annet enn driftsinformasjon. Instituttet var svært positivt innstilt til arrangementet fra starten av, og hjalp til med både økonomien og blestingen.

Resten er historie, som man sier. Hadde vi hatt noen anelse om hvor enorm arbeidsmengde prosjektet skulle kreve, hadde vi nok kuttet kraftig ned på ambisjonene (skjønt underestimering er jo gammel tradisjon i IT-relaterte prosjekter). Det var nok ingen i den første Dagen@Ifi-komitéen som fikk produsert noe særlig vekttall i oktober det året, og mange endte opp med å «døgne» på Ifi en god del (men sosialt var det!). Resultatet gikk imidlertid over all forventning - vi var i utgangspunktet bekymret for om vi ville greie å samle så mye som 100 studenter, men folk strømmet på hele dagen og ble værende på Ifi til langt utpå kvelden. Det førte faktisk til at vi gikk tomme for øl (!) relativt tidlig på kvelden, og det var bare takket være heroisk innsats fra blant andre Dag-Erling Smørgrav at vi fikk inn nye forsyninger utpå kvelden og dermed unngikk «katastrofe».

Det ble ikke noen sammenslåing av andre foreninger enn Cyb og Verdande, men til gjengjeld oppnådde man hensikten med prosjektet: det ble vesentlig mer samarbeid og kommunikasjon mellom foreningene etter dette særdeles vellykkede opplegget.

Det at den første Dagen@Ifi fikk så mye oppmerksomhet og ble en slik suksess, gjorde det lett å rekruttere både styremedlemmer og funksjonærer året etter, og siden 2003 har arrangementet blitt større og mer veldrevet for hvert år. Undertegnede «dinosaur» vil gjerne sende en stor takk til alle dere som har ofret tid, krefter, vekttall og studiepoeng for Dagen - slitsomt er det, men også helt fantastisk når det går så bra som det gjør!

\subsection{... og litt om Navet}
av Anna Dahl

Navet har sitt opphav både i dagen@ifi og i bedriftskomitéen i Cyb. I mange år arrangerte Cyb næringslivsdag, og idéen til dagen@ifi sprang delvis ut fra et ønske om å stable noe sånt på beina igjen, i tillegg til å ta opp tradisjonen med foajéfest - også noe Cyb vanligvis hadde arrangert.

I 2006 hadde IT-bransjen hentet seg opp igjen etter det store .com-sprekket, og dagen@ifi gjorde det svært godt økonomisk - næringslivets interesse for å profilere seg overfor studentene bare økte. Cyb fikk en del henvendelser fra ulike bedrifter videresendt fra instituttet, og enkelte kom direkte fra bedrifter der noen husket at de hadde vært «bedriftsmedlemmer» av Cyb noen år tilbake. Det var pinlig åpenbart at Cyb, hvis bedriftskomité i praksis lå brakk, ikke hadde noe godt mottaksapparat for denne typen henvendelser. I tillegg hadde flere av foreningene sporadiske arrangementer med forskjellige bedrifter, og behovet for koordinasjon begynte å oppstå.

Daværende bedriftsansvarlig i Cyb, undertegnede, kom på bakgrunn av dette frem til at en eventuell gjenoppliving av bedriftskomitéen måtte bli nok et samarbeidsprosjekt mellom foreningene. Det var forbausende lett å rekruttere oppegående foreningsmennesker til dette prosjektet. Ettersom det ville være en del penger involvert, ble det besluttet å opprette en egen forening med tilhørende statutter, konti etc. Det var også åpenbart at man ville få behov for en mekanisme som kunne sørge for at disse midlene kom samtlige Ifi-studenter til gode, og et eget utvalg - fordelingsutvalget - ble opprettet med dette formålet.

Oppstartsmøtet ble holdt 16. februar 2006, med to representanter fra Cyb (Anne Marie Bekk og undertegnede), to fra Mikro (Håkon Olafsen og Håvard Pedersen) og to fra ProsIT (Christian Mikalsen og Tommy Gudmundsen). Blant pionérene var også Geir Nilsen og Magne Eimot. I ekte gründer-ånd startet man med å arbeide seg frem til et navn, en logo og en visuell profil, og fikk opp en webside. Ettersom foreningen skulle fungere som et sentralt kontaktpunkt mellom næringslivet, studentene og i noen grad instituttet, dukket begrepet «hub» forholdsvis raskt opp, og dette ble oversatt til Navet. Etter noen måneder viste det seg at vi ikke hadde vært helt alene om disse tankene, da den nye etaten NAV ble lansert - men da var det allerede blitt i overkant ressurskrevende å endre det.

Med få endringer er informasjonsteksten om Navet den samme som i oppstarten:

«Navet er bedriftskontakten ved Institutt for informatikk ved Universitetet i Oslo. Hensikten med Navet er å gjøre det enkelt for bedrifter å komme i kontakt med studentene ved instituttet, ved å tilby:

\begin{itemize}
\item et sentralt kontakt- og koordineringspunkt for alle bedriftsrelaterte aktiviteter ved instituttet.
\item praktisk hjelp ved bedriftspresentasjoner og andre typer arrangmenter (romreservasjon, plakatopphenging, utsendelse av SMS mm.).
\item oversikt over bedriftsrelaterte aktiviteter for studenter.  
\end{itemize}

Engasjerte studenter tok initiativet til å starte bedriftskontakten, og Navet er studentdrevet.»

Høsten 2006 kom det inn nye friske krefter i tillegg - Are Wold, Daniel Chaibi, Fredrik Klingenberg og Magnus Korvald - og aktivitetsnivået økte. Det første store arrangementet med en enkeltbedrift var da Google kom og holdt foredrag 5. oktober. Den nyoppstartede norske avdelingen i Trondheim stilte opp med flere representanter, og det var enorm interesse: Store Auditorium ble smekkfullt, folk måtte sitte i trappene for å få plass, og en del måtte snu i døren. Sluttfakturaen for mat og drikke kom på oppunder 22 000 kroner (!).

Navet-styret hadde ikke formelle roller i starten, men da Daniel Chaibi ble valgt til leder i 2007 økte aktivitetsnivået igjen betraktelig - særlig promoveringen ble det virkelig fart på. I årene som har gått siden da har Navet utviklet seg til å bli en svært profesjonell og veldrevet forening, og oppfyller i aller høyeste grad sitt formål om å være et kontaktpunkt mellom Ifi-studentene og næringslivet.

\subsection{De mørke årene}
av Ole Kristian Hustad

Etter 2003 har det vært en markant nedgang i studentaktiviteten generelt i mange foreninger, kvalitetsreformen har fått mye av skylden for dette, men undertegnede mener å kunne observere at denne nedgangen begynte litt tidligere (allerede 2001). For informatikkerne sin del var det også en splittelse blandt studenter som følge av nye “linje-foreninger” som kom med studielinjene som ble opprettet i forbindelse med kvalitetsreformen. Der man tidligere kun hadde informatikerer og foreningene;  “Cyb, Ping og Verdande” fikk man, også “mikro”, “Prosit”,  “TOOL” og “ITSLP” som hver og en hadde egne “arrangører” for deres linje. Det har vært mange ildsjeler blant informatikerene opp igjennom årene, men desverre ikke nok til å holde liv i alle sammen hele tiden.

I flere års tid prøvde styret i Cybernetisk selskab å øke aktivitetsnivået og vitaliteten i foreningen, og på tross av mange tapre forsøk og hederlig innsats falt alt fra hverandre i 2008 og Cyb lå nede for telling i nesten ett helt år uten aktivitet. Men sent på høsten 2008 ble det tent en liten gnist og nye og gamle medlemmer deltok i ett nytt styre ledet av Geir Arild Byberg (geiraby), og senere tok Øyvind Bakkeli (oyvindbak) over.

\subsection{Men bak skyene skinner solen}
av Magnus Johansen og Ole Kristian Hustad

Gjenopplivningen begynte så smått med en serie av mindre arrangementer som for eksempel filmkvelder og pokeraftener, men etter hvert begynte styret å planlegge større ting, samt adoptere arrangementer fra andre foreninger som hadde gått under. Det første store, egne arrangementet til Cybernetisk Selskab i denne perioden var fagdagen That's IT, et faglig alternativ til dagen@ifis næringslivsrettede heldagsarrangement. Dette ble satt opp i lag med Realistforeningens 150års jubileum ved Margrete Raaum. Under denne tiden fikk Cybernetisk Selskab som en oppadstormende forening sin første tunglærte lekse ved drift av forening. Lederne i foreningen tok på seg mesteparten av arbeidet selv og resultatet var at Geir ble utbrent og måtte trekke seg fra sin stilling noe senere.

Med en ny leder Øyvind Bakkeli på plass og noe nytt blod i styret tok Cybernetisk Selskab et nytt semester på strak arm og begynte straks å planlegge noen større arrangementer. Det ble forespurt styret om de ønsket å ta over Hemsedal turen som tidligere har vært arrangert av den døde foreningen PROSIT, noe styret behandlet i god tid og senere gikk med på å støtte etter noe diskusjon. Denne turen ble gjennomført med CYB som sponsor og er sett på som en dundrende suksess og den første markeringen som viser at CYB var kommet tilbake og var i stand til å fungere videre. Underveis vokste også medlemsmassen og foreningen fikk hele tiden flere interessenter som ville jobbe med dem. I denne perioden fikk også Cybernetisk Selskab tildelt driftsansvaret for den kommende studentpuben i det nye Informatikk bygget, da dubbet IFI2. Magnus Johansen som da satt sitt første semester, ønsket seg umiddelbart til vervet som leder for denne puben og har siden da jobbet med dette.

I denne perioden jobbet CYB videre med sine arrangementer og prøvde seg på nye konsepter underveis, blant annet ble det arrangert LAN som en nysatsning, en tradisjon som har blitt holdt og vist seg å være meget populær. CYB vokser i denne tiden videre og blir bare større og større.

Tiden faller på, nye semestre kommer og CYB fortsetter å arrangere sine velkjente arrangementer. Foreningen trer inn i en periode som nesten kan late til å være stille, men bak teppet foregår det mye som ikke syns for allmennheten. I forberedelse til en ny æra som forening, med egen studentkjeller og økt aktivitet som kommer fortere enn foreningen hadde tenkt seg, jobber styret frenetisk med å oppdatere sitt lovverk samt legge til rette for den økte aktiviteten. Styret ser seg ut en ny måte å arrangere ting på, og nye verv opprettes, samt en ny webside blir til. Et kjellerstyre opprettes under Kjellermester Magnus Johansen, og driftsarbeidet fases ut fra Hovedstyret slik at administrative oppgaver kan gis mer oppmerksomhet. Det prosjekteres i denne perioden inn et arrangementsstyre som kan ta seg av jobbingen med arrangementer, samt åpne for mer deltagelse fra studentmassen som har ideer.

I ettertid kan vi med fasit i hånden si at gjenopplivningsforsøket ble en braksuksess. Engasjerte unge Cybbere med mye engasjement og med god drahjelp av den kommende “IFI-kjelleren” i det nye “IFI-bygget” (Ole Johan Dahls Hus, bedre kjent som IFI2) har gitt foreningen en ny vår.

%TODO bilde, utenfor Escape

Studentkjelleren har fått navnet “Escape” og Cybernetisk Selskab har offentliggjort navnet på vår høyeste beskytter Aptenodytes forsteri (keiserpingvinen) høst 2010. 

%TODO bilde, fortsatt utenfor Escape (men nå med kø)
%TODO bilde, Morten Dæhlen og Magnus Johansen i Escape 13. mai 2011

I nåtiden har CYB fått på plass sitt arrangementsstyre og det jobbes nå med alle arrangementer som skal skje. Foreningen venter nå i spenning på å få åpne sin studentpub, som har fått navnet Escape og ber til sin øvre beskytter Keiserpingvinen (Aptenodytes forsteri) om gode stunder framover. Under ledelse av Marius Næss Olsen fortsetter Cybernetisk Selskab sin vekst, samtidig som de ser tilbake på sin fortid som en underdog. De husker sine forgangere som har blitt med i veksten og bidratt med sin visdom og kunnskap, og ser frem mot en tid der hvor foreningen nyter godt av lidenskapelige studenters driv til å gjøre noe mer for sin egen og andres studiehverdag, og ser aldri seg selv nede i bunnen igjen.
\end{document}