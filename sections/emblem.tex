\documentclass[../main.tex]{subfiles}

\begin{document}
av Rolf Bjerknes

I den høytidsstemning som føles ved feiringen av det 25 årige jubileum til den ærverdige forening, Cybernetisk Selskab, faller det naturlig å dvele ved Selskabets emblem. Som den intellektuelt skolerte leser ved dyktig observasjon sikkert allerede har registrert, inneholder nevnte emblem en rekke viktige komponenter relatert til Selskabets fødsel og misjon. Den da helt nystiftede forening, med formann Ivar Jardar Aasen, utlyste allerede våren 1969 en konkurranse for om mulig å få fram forslag til et emblem som kunne være Selskabet verdig. Emblemet burde kunne lokke fram assosiasjoner om Selskabets formål. Det kom inn noen forslag, kanskje færre enn ønsket. Korrelasjonen mellom det å studere realfag og kunstneriske evner var dengang muligens noe svak. Dette er som vi vet, ikke tilfelle i dag. Imidlertid sendte daværende stud.real. Rolf Lind inn fire forslag, datert 22. mai 1969 (De mange hull, Det gode selskab, Pilen og Negativ feedback) og Karl P. Fischer ett forslag. Styret for høst-semesteret 1969, med formann Ole-Herman Bjor, fungerte som jury. I et brev til Rolf Lind, datert 29. august 1969, uttrykker juryen sin preferanse for det ene forslaget. Samtidig tillater juryen seg å foreslå noen modifikasjoner: Bokstavene gjøres større og plasseres utenfor kulene, og hele emblemet avgrenses med en superellipse. Rolf Lind sa seg helt enig i dette, slik at den endelige versjonen ble slik som vist foran. Rolf Lind er cand.real. 1971, Fysikk hovedfag, linje for kybernetikk, og er nå en verdifull medarbeider hos IBM. Karl Petter Fischer er cand.real. 1972, Kjemi hovedfag, og er nå bosatt i Sandefjord. Lykkeligvis er også de andre innsendte forslag bevart i Selskabets arkiver, og alle fem er gjengitt nedenfor i sin opprinnelige form.

Det ferdige emblemet er bygget opp av tre komponenter:
\begin{enumerate}
\item Superellipsen, Piet Hein 1960
\item Cybernetics - CYB, Norbert Wiener 1948
\item Sentrifugalregulatoren, Christian Huygens 1657
\end{enumerate}
Disse tre vil bli nærmere beskrevet i det følgende.

%TODO ta med bilder av de innsendte forslag til emblemkonkuransen

\subsection{Superellipsen}
Dansken Piet Hein (f. 1905) er kanskje mest kjent for sine `Gruk' under pseudonymet `Kumbel'. Han `oppfant' superellipsen i julen 1959. Dette er beskrevet i boken: Dobbeltmasken, Piet Hein 75 år, København 1980. Den ble tatt i bruk for å utforme en avlang rundkjøring på Sergel's torg i Stockholm. Senere ble Olympiastadion i Mexico City utformet på samme måte, den sto ferdig til Olympiaden i 1968. En skjønner nå at emblemets ramme ble utformet ifølge ideer som var høyst aktuelle på den tiden. Piet Hein brukte den generelle formelen

%TODO her skal det være en formel, men jeg klarer ikke sikkert å si hvordan formelen i Word-dokumentet er ment å skrives
% kan det være p/(x/a)+p/(y/b) = 1? nei, det er umulig å si
% kanskje det står i 25-årsheftet?

hvor han valgte eksponenten p = 2.5. Verdien p = 2.0 gir den vanlige ellipsen. Økende verdier av eksponenten gjør figuren mer rektangulær. For rammen rundt CYB's emblem er det brukt verdiene a = 3 og b = 4. Piet Heins genistrek ligger ikke spesielt i å velge den eksakte tallverdien p = 2.5, men heller i det å utforme det fengende navnet `Superellipsen'. Danske møbelprodusenter lanserte straks salongbord med samme fasong. En kunne også få kjøpt `Superegg', både av messing og av sølv. Disse har den egenskap at de kan stå på enden, i motsetning til et vanlig egg, og et slikt egg ville vært et funn for Christofer Columbus. Matematisk sett får `egget' denne egenskapen straks eksponenten er større enn 2.0, men i praksis må den være vesentlig større for at det skal ha noen demonstrasjons- effekt.

\subsection{Cybernetics}
Den amerikanske matematikeren Norbert Wiener (1894-1964) skrev i 1948 boken: CYBERNETICS or control and communication in the animal and the machine. Her lanserte han `Kybernetikk' som en egen vitenskap. Selve ordet dannet han fra det greske ordet for `styrmann': kybernetes, slik at kybernetikk må bety `styrmannskunst'. Ordet `guvernør' skal ha samme språklige rot, og altså også den engelske betegnelsen for sentrifugalregulator: Governor. Norbert Wiener er også kjent for boken: Extrapolation, Interpolation, and Smoothing of Stationary Time Series. Denne boken kom ut i 1942, men ble da klassifisert som militær hemmelighet, slik at den ble først kjent da den ble utgitt på MIT Press i 1949. Begge disse bøkene gjorde stort inntrykk, også her i Norge. Ordet `Cybernetics' var altså et typisk `moteord' i 60-årene, og det forklarer foreningens fulle navn, mens man i daglig tale oftest bruker den korte formen `CYB'. Nå i 90-årene har ordet dukket opp igjen, nå som: `Cyberspace'. Norbert Wiener introduserte også Wiener-filteret som er et optimalt filter for lineære stasjonære systemer. Wiener-filteret kan ses på som en forløper for Kalman-filteret, både teknisk og historisk, fordi Kalman-filteret under stasjonære forhold er ekvivalent med et Wiener-filter.

\subsection{Sentrifugalregulatoren}
Nederlenderen Christiaan Huygens (1629-1695) er kanskje mest kjent for sin bølgemekanikk: Huygens' prinsipp. Han arbeidet også meget med å konstruere en nøyaktig tidsmåler. Han oppfant pendeluret i 1656, men han prøvde også flere andre mekanismer, blant annet sentrifugalregulatoren som han oppfant i 1657. Briten James Watt har i ettertid feilaktig fått æren av dette. Han fikk nemlig patent i 1788 på å anvende den til hastighetregulering av dampmaskinen. En annen brite, Thomas Mead, fikk i 1787 patent på å anvende sentrifugalregulatoren til å regulere avstanden mellom møllestenene i en vindmølle. Den ble også brukt til å regulere seilene på vingene til vindmøllen for å få møllen til å gå med jevn hastighet. Nedenfor er det vist noen eksempler på hvordan sentrifugalregulatoren ble utformet. Den første grundige matematiske analysen av en slik reguleringsmekanisme ble levert av den kjente britiske vitenskapsmannen James Clerk Maxwell (1831- 1879) med den klassiske publikasjonen av Maxwell, J. C.: On Governors, Proc. Roy. Soc. (London), 16, 1868.

%TODO bilde, vindmølle
%TODO bilde, en sentrifugalregulator som regulerer en dampmaskin

Som man skjønner ligger det megen omtanke bak emblemet til Cybernetisk Selskab. Det kan derfor kanskje passe å avslutte med et Gruk:
\begin{center}
Tæk ditt tag\\
med vid og viden\\
Ånd alene\\
trodser tiden.
\end{center}
\end{document}