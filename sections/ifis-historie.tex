\documentclass[../main.tex]{subfiles}

\begin{document}
av Narve Trædal

Da fakultetsrådet ved mat.nat.fakultetet i møte 4. desember 1975 besluttet å opprette Institutt for informatikk fra 1. januar 1977, betydde dette at undervisningen i data-fag ved fakultetet endelig hadde fått en felles organisatorisk basis. Hvorvidt det også betydde en felles faglig basis, er vel et mer diskutabelt spørsmål. Men i alle fall fikk de tre faggruppene som instituttet besto av, databehandling, numerisk analyse og kybernetikk, et helhetlig studieopplegg og langt på veg en felles studentmasse å relatere sin aktivitet til.

Men selv om faget nå tilsynelatende framsto som et nyskapt universitetsfag, var realiteten den at deler av faget alt hadde en over 20 år gammel historie på fakultetet.

\subsection{Røttene}
Lenge var datafaget en aktivitet for spesielt interesserte, fra en sped begynnelse i første halvdel av 50-årene ved Fysisk institutt, hvor den "hjemmelagde" datamaskinen "Nusse" ble tatt i bruk i 1953. Maskinen tilhørte egentlig Sentralinstitutt for industriell forskning, SI, som leide rom i kjelleren i Fysikkbygget, men ble også brukt av universitetet. Ved Matematisk institutt er det rimelig å betrakte professor Selmers seminar på midten av 50-tallet som et startpunkt. Den anvendte instituttsektoren var også tidlig inne i bildet, bl.a. holdt Harald Keilhau ved Forsvarets forskningsinstitutt, FFI, kurs i programmering i 1958. Opptakten til EDB-senteret (det nåværende USIT) kom litt senere, ved at universitetet kjøpte inn en Wegematic 1000 i 1960.

Det var innen disiplinene matematikk og fysikk/ingeniørfag at røttene lå. Den tredje komponent som senere ble konstituerende for fagtilbudet ved instituttet, organisasjons- og administrasjonskunnskap, var dårlig representert ved UiO.

\subsection{Universitetsdisiplin eller redskapsfag}
På slutten av 60-tallet og utover i 70-årene var det nærmest en eksplosjonsartet økning i interessen for EDB-utdanning fra studentenes side. Dette var selvsagt et uttrykk for at interessen i samfunnet for dette feltet øket sterkt. I alle offentlige utredninger om teknologisk satsing fra 1965 og framover, står data-området sentralt.

Når det gjaldt akademisk, forskningsbasert utdanning, var imidlertid interessen i politiske kretser mindre. Det ble i det alt vesentlige fokusert på kortvarig redskapspreget utdanning. Det førte til at universitetetene (og NTH) langt på veg ble stående alene om å se nødvendigheten av at det nye faget ble gjort til gjenstand for eksperimentell naturvitenskapelig forskning.

Dette var i og for seg naturlig. Den utdanningspolitiske dagsorden i slutten av 60-årene og framover var i det vesentlige preget av utredninger knyttet til etablering av et nytt distriktshøgskolesystem. Ottosen-komiteen la opp til at den framtidige satsingen på postgymnasial utdanning skulle skje i distriktene, ved etablering av to- og tre-årige yrkesrettede utdanninger. Dette ble fulgt opp av regjering og Storting. I DH-konseptet hadde dataundervisning en sentral plass, men vesentlig som redskapsfag innen studieretninger for økonomi og administrasjon. Bare ved Molde, Østfold og Agder DH ble det etablert et 2-årig spesialstudium i EDB. 

I disse årene foregikk også en kraftig opprustning av den lavere og midlere tekniske utdanningen. Den 2-årige ingeniørhøgskolen ble normen. Mange steder gikk ingeniør-utdanningen inn som en del av distriktshøgskolene. I tråd med Ottosen-komiteens innstillinger ble ressursene kanalisert inn i denne storstilte satsningen på kortere desentralisert utdanning.

\subsection{Situasjonen blir uholdbar}
I de siste 8-10 år før instituttstiftelsen aksellererte interessen blant studentene for fakultetets datatilbud år for år. Dette skapte store problemer for flere institutter, særlig Matematisk institutt, avdeling D, men også for linjene for kybernetikk og delvis elektronikk ved Fysisk institutt. Tilstrømningen til mat.nat.-fakultetet forøvrig var i begynnelsen av 70-årene relativt moderat, særlig sammenlignet med resten av universitetet, som også opplevde en studentboom. Når så et relativt marginalt område ved fakultetet, som data-fagene i realiteten var, fikk en så stor etterspørsel, ble det raskt en sterk ubalanse i undervisnings- og veiledningsbelastningen. De ansatte ved de andre avdelingene ved Matematisk institutt, og storparten av de andre instituttene ved fakultetet, hadde relativt rolige tider, mens deres kolleger ved avdeling D fikk hendene så fulle med utarbeidelse av undervisningsmateriell, undervisning og veiledning, at det ble omlag umulig å få tid til forskningsrelaterte aktiviteter. Særlig gjaldt dette databehandlerne. Og de som i første rekke måtte ri av stormen i begynnelsen av 70-årene var først og fremst professor Ole-Johan Dahl, sammen med universitetslektorene Arne Jonassen og Olav Dahl. Fagretningen for numerisk analyse var ikke fullt så etterspurt.

Ved Fysisk institutt var det lignende forhold. Studentinteressen for kybernetikk var stor. Instituttet befant seg på slutten av 60-tallet plutselig i en situasjon der en stor del av studentene ønsket hovedfag i en fagretning hvor det ikke fantes undervisningstilbud! (For en nærmere beskrivelse av dette henvises til artikkelen om Cybernetisk Selskaps fødsel.) De stillingene som ble opprettet for Lars Walløe, Ellen Hisdal og Rolf Bjerknes, kom som et svar på dette presset. Elektronikk-linjen var også utsatt, men ikke i samme grad som kybernetikk-miljøet.

\subsection{Forløpet til instituttstiftelsen}
Dramatikken i denne situasjonen ble for avdeling Ds vedkommende beskrevet i en utredning som ble utarbeidet av alle tilsatte ved avdelingen. Den fikk det malende navnet "Gjøkungen", og var et vel dokumentert nødsskrik, hovedsakelig formulert av avdelings-bestyreren, universitetslektor Arne Jonassen. Det er vel ikke urimelig å betrakte den datoen innstillingen ble lagt fram: 9. mars 1974, som unnfangelsesøyeblikket for instituttet, selv om utredningen ikke konkluderte sterkere enn at fakultetet i nær framtid burde vurdere organiseringen av informatikkens administrative plassering på lang sikt. Som man ser, var her informatikk brukt som et samlebegrep for den datarelaterte undervisningen ved fakultetet. I følge utredningen var det i tråd med hva som var vanlig internasjonalt, særlig i Europa.

"Gjøkungen" resulterte i at fakultetet satte ned en komité "for å vurdere datafagenes ressursmessige stilling og administrative plassering ved fakultetet". Innstillingen fra Informatikk-komitéen, som den ble kalt, kom i juni 1975, og konkluderte enstemmig med at det burde opprettes et nytt institutt bestående av "numerisk matematikk, databehandling, kybernetikk og digitalteknikk. Derimot så ikke komiteen noe behov for "administrativ databehandling", som komiteen mente var dekket andre steder, bl.a. i Bergen (Handelshøyskolen og Institutt for informasjonsvitenskap).

Informatikk-komiteen ble fulgt opp av utredninger om geografisk samling, og forslag til ny studieplan, og i desember 1975 kunne fakultetet fatte vedtak om instituttstiftelsen med virkning fra 1. januar 1977.

\subsection{Stillingsressursene}
Ressurssituasjonen var i denne "svangerskaps"-tiden, såvel som i tiden etter instituttfødselen, fortsatt mager. Informatikk-komiteen hadde konkludert med at et institutt ville ha behov for 29 vitenskapelige stillinger (inklusive 5 II-stillinger) og 3 administrative stillinger. Instituttets behov for teknisk assistanse ble det antatt kunne dekkes av EDB-senteret, samt av 2 rekrutteringsstillinger (vitenskapelige assistenter). Den faktiske situasjonen var imidlertid at miljøene som var aktuelle i instituttet bare disponerte 17 vitenskapelige stillinger (inklusive 2 II-stillinger), 1 kontorstilling og ingen tekniske vit.ass.-stillinger.

Selv om alle syntes sympatisk innstilt til det nye instituttet, så var det altså et stort gap mellom det behovet som ble anslått, og de stillinger som var tilgjengelig. Øremerkede ressurser over statsbudsjettet forekom nesten ikke. Det var stillingsstopp til UiO. De stillinger som ble tilført det nye instituttet, var derfor kun de stillinger som var besatt av de vitenskapelig ansatte som ble flyttet fra Matematisk institutt (avdeling D ble i sin helhet overflyttet) i tillegg til kybernetikkgruppen fra Fysisk institutt.

Omlag alle ressurser måtte altså tas ved intern omrokering av fakultetets eksisterende ressurser - og det er som kjent alltid en tung prosess. Fakultetets dekanus, Tore Olsen, var imidlertid svært innstilt på at prosessen skulle lykkes. Som professor i elektronikk og tidligere bestyrer ved Fysisk institutt hadde han første hånds kjennskap til problemene der, og klarte å få instituttet til å avgi ressurser, sammen med sin fagretning for kybernetikk. Mikroelektronikkmiljøet ved elektronikklinjen ble beholdt ved Fysisk institutt, selv om det ble understreket at digitalteknikk var et naturlig interessefelt for det nye instituttet.

Et særegent problem var de ikke-vitenskaplige stillingene. Et eget institutt forutsatte egen administrasjon og egen teknisk stab. Administrasjonen besto fra starten av en kontorstilling som ble overført sammen med avdeling D, og av instituttsekretær Elisabeth Hurlen som ble nyansatt i halv stilling.

En annen årsak til at det nye instituttet ikke fikk tilført flere stillinger, var at det i årene rundt instituttstiftelsen var tegn som tydet på at studenttilstrømningen ville flate ut. Mange dro derfor raskt den konklusjonen at interessen for data i ungdomsmassen hadde kuliminert. Dette viste seg å være en sterkt forhastet konklusjon. Studentpresset økte raskt til nye høyder. Instituttet styrket stadig sin stilling som det matnat-institutt som hadde det suverent verste tallmessige forholdet mellom lærere og studenter. Selv om instituttet som nevnt møtte en betydelig velvilje i fakultetsledelsen, var det likevel begrenset hva fakultetet kunne bidra med. Likevel øket tallet på ansatte jevnt og sikkert. Ti år etter instituttstiftelsen hadde instituttet kommet opp i 49,5 stilling, dvs. en økning på 30 siden starten. Over halvparten av disse stillingene var blitt tilført via omdisponering på fakultetet. I 1979 hadde vedtatt et "Program for styrking av fagområdet informatikk" der man gikk inn for en fordobling av instituttets utdanningskapasitet. Programmets målsetting, både med hensyn til antall nye stillinger og utdanningskapasitet, ble oppnådd, men noen bedring i arbeidsforholdene for de ansatte var ikke oppnådd. Fakultetet vedtok et nytt program høsten 1984, "Program for videre utbygging av fagområdet informatikk", hvor målsettingen eksplisitt ble satt til en fordobling av antall ansatte ved instituttet. På grunn av knapphetsfaktorer, både hva angikk stillingsressurser og antatt antall kvalifiserte søkere, ble det sagt at det ikke var realistisk å klare mer enn halvparten av denne fordoblingen innen 1990. Det så således ikke lyst ut for en rask forbedring av arbeidsforholdene.

\subsection{Utstyr}
Tekniske stillinger ble ikke ansett som nødvendig for det nye instituttet. EDB-senteret hadde hele tiden stått for maskinutrustningen, både til studenter og forskere. Ressurssituasjonen ikke slik at det kunne være på tale å bygge opp en egen maskinpark for instituttet. EDB-senteret i 70-årene tiden ytte en betydelig bistand, både teknisk og faglig, ved å stå for mye av hovedfagsveiledningen ved instituttet. Da tilstrømningen økte, og det ble opprettet en egen terminalstue for laveregrads studenter i EDB-senterets regi, samtidig som kravene til EDB-senterets virksomhet fra resten av universitetet økte, hendte det at samarbeidsklimaet til tider ble lett anspent. Informatikkmiljøet hadde av og til følelsen av å ikke bli prioritert med sine behov. Det verserer fortsatt historier om at hullkortbunkene til Ifi-ansatte hadde lett for å havne i gulvet på EDB-senteret, dersom man ikke hadde den rette holdningen til de maskinansvarlige. Slike ekstreme hendelser var vel ikke dagligdagse, men det var nok naturlig at interessene til de to datamiljøene skilte lag, etter hvert som kravene fra omverdenen til de to miljøene økte.

Utviklingen av instituttets egen maskinpark og nett skjedde først fra 1980, da Tor Sverre Lande ble ansatt i en amanuensis-stilling. Han hadde i disse årene nærmest eneansvaret for den tekniske kompetanse. Ut over på 80-tallet oppsto det spørsmål om hvilken strategisk utstyrspolitikk instituttet skulle legge seg på. Instituttet samlet seg om en politikk som bygde på distribuerte løsninger med arbeidsstasjoner og servere, basert på programvare som skulle gjøre instituttet i størst mulig grad uavhengig av enkelte maskinleverandører. Mot dette synet sto en annen linje, som langt på veg var den rådende ellers i dataverdenen, nemlig å satse på store sentrale maskiner dominert av en enkelt utstyrsleverandør. EDB-senteret var på denne tiden representant for en slik politikk, som også passet godt inn strategien til f.eks. Norsk Data.

Da instituttet i 1982 ble tilkoblet Internett og visst nok som den første i Norge tok i bruk Berkeley UNIX, gikk det således mot strømmen. Utviklingen senere har vist at det var en meget framsynt linje, som i dag har fått alminnelig oppslutning, både nasjonalt og internasjonalt.

%TODO tabell, utstyr

\subsection{Faggruppene}
Fra starten av satset det nye instituttet altså på numerisk matematikk, databehandling og kybernetikk, med databehandling og kybernetikk som de særlig populære feltene, sett fra studentsynspunkt. Men fagspekteret ble fort utvidet. Selv om informatikk-komiteen hadde avvist behovet for administrativ databehandling, hadde det innen avdeling D eksistert et hovedfagskurs IN 60, som omhandlet samfunnsmessige aspekter ved bruk av databehandling. Dette var en ny og original innfallsvinkel til informatikken, hvor hovedvekten ble lagt på systemutviklingen, sett i relasjon til de sosiale omgivelsene systemene skulle brukes i. Emnet ble undervist av eksterne krefter, dvs. i hovedsak av forskningssjef Kristen Nygaard ved Norsk Regnesentral, NR. Fra 1.4.1977 ble han imidlertid ansatt som professor II, og rundt ham ble den undervisningen og forskning som senere ble konstituerende for faggruppen for systemarbeid, organisert.

I motsetning til den systemarbeidsrelaterte aktiviteten, hadde informatikk-komiteen sett undervisning og forskning i digitalteknikk som et naturlig satsingsområde for et nytt institutt. Men digitalteknikkaktiviteten ble som nevnt ikke overflyttet til det nye instituttet. Det ble imidlertid, i samarbeid med Fysisk institutt, arrangert kurs i digitalteknikk både på lavere og høyere nivå som et ledd i informatikkstudiet. Forsker I ved FFI, Yngvar Lundh ble knyttet til det nye instituttet som professor II fra 1.10.1980. Fra den tid hadde digitalteknikkmiljøet et stillingsmessig fotfeste innad på instituttet, og faggruppen for digitalteknikk ble litt om senn bygget om rundt ham. Ansettelsen skjedde ikke uten sverdslag. I instituttstyret ble det av et mindretall stilt spørsmålstegn ved om instituttet ønsket å ha et så tett samarbeid med en institusjon som vesentlig drev med forskning på våpensystemer og annen militær teknologi.

Ved jubileumstidspunktet har instituttet 4 faggrupper, i og med at faggruppene for numerisk analyse og kybernetikk i 1990 ble slått sammen til en faggruppe for matematisk modellering. Det har altså gått slik at i en tid der tendensen til differensiering og oppsplitting innen vitenskapelige disipliner er sterk, så har disse to gruppene, som opprinnelig kom fra hvert sitt institutt, kunnet gå sammen om ansvaret for en studieretning innen informatikken.

På den andre siden har det også foregått knoppskyting, særlig fra det matematiske modelleringsmiljøet. Bildebehandling, som kan regne sine røtter tilbake til ansettelsen av Fritz Albregtsen i en NAVF-finansiert laboratorieingeniørstilling i 1983, ser ut til å utvikle seg i retning av å bli en egen faggruppe. I 1993 ble et eget hovedfag i anvendt og industriell matematikk etablert. Hovedfaget er i hovedsak et samarbeidsprosjekt med Institutt for matematikk, men også enkeltmiljøer ved Kjellerinstituttene er representert.

\subsection{Lokalitetene}
Lokalmessig var situasjonen for det nye instituttet relativt kummerlig. I startfasen hadde instituttet lokaler i Matematikkbygningen. Administrasjonen og faggruppen for kybernetikk, som var flyttet fra Fysikkbygningen, hadde lokaler i 5. etasje, mens databehandlere og numerikere stort sett beholdt sine gamle kontorer, og befant seg således marmorert inn i Matematisk institutts arealer. I 1980 ble instituttet flyttet til Fysikk-bygningen, og fikk lokaler i Østfløyen. Dette var et framskritt, sett fra et samlingssynspunkt, men heller ikke denne situasjonen var tilfredsstillende. Mye tid gikk med til å løse romproblemer, ofte på bekostning av Fysisk institutt.

I begynnelsen av 80-årene dukket ideen om et eget informatikkbygg opp. En sentral person i dette arbeidet var Arne Jonassen, som nå var ansatt i NR. Da så NTNF fattet interesse for planene, ble det fart i prosessen, og NTNF sto som byggherre. Ifi og NR flyttet inn sommeren 1988. Selv om det også i det nye huset relativt fort meldte seg ombyggingsbehov, og kapasiteten var sprengt nærmest før innflytting, så var altså nå instituttet for første gang herre i eget hus.

\subsection{Det lysner på ressurssiden - Informasjonsteknologiprogrammet}
Lokaliseringen til Informatikkbygget står som en synlig milepæl i instituttets historie. For faget var nok innføringen av det nasjonale informasjonsteknologiprogrammet fra og med budsjettåret 1987 viktigere. Sammenfallet mellom disse to begivenheter gjør at det er naturlig å se årene 1987 og 1988 som et epokeskille i instituttets historie.
Det regjeringsinitierte Informasjonsteknologiprogrammet som kom i 1987 representerte et kvalitativt sprang i positiv retning. For 1987 ble det bevilget 11,5 mill. kr., noe som betydde er firedobling av instituttets midler til drift og innkjøp, og gjorde instituttet i stand til å foreta en kraftig opprustning av såvel terminalstuer som arbeids-plasser. Selv om IT-bevilgningene gradvis er blitt redusert, og nå langt på veg er gått inn som en regulær del av instituttets driftsmidler, så har konsekvensene vært at situasjonen, særlig på utstyrsområdet, er blitt betydelig bedret. Nå må situasjonen når det gjelder driftsressurser karakteriseres som rimelig god, sammenlignet med mange andre institutter. Det utstyret instituttet har til rådighet er fullt på høyde med det som finnes ved lignende institusjoner internasjonalt. Det samme må sies om situasjonen når det gjelder midler til faglige og vitenskapelige reiser, hvor ressurssituasjonen ved instituttet trolig er meget bra, sammenlignet med andre institutter ved universitetet.

Informasjonsteknologiprogrammet betydde også flere stillinger, først midlertidige undervisningsstillinger, som senere er blitt gjort faste. I samme tidsrom har også stillingstilførselen over budsjettene økt, særlig via de siste årenes diverse mer kortsiktige bevilgninger for å øke studentopptaket og bedre gjennomstrømningen. I tillegg har instituttet de siste årene opprettet en rekke deltidsengasjementer som amanuensis II. Tallet på rekrutteringsstillinger ved instituttet er fortsatt lavt, men bl.a. gjennom strategiske teknologiprogrammer finansiert gjennom NFR, har instituttet de siste årene fått opprettet flere blokkstipendiatstillinger.

\subsection{Samarbeid med instituttsektoren}
Som en måte å overleve og utvikle seg på, både faglig og undervisningsmessig, har det hele tiden vært et særpreg ved datamiljøene ved fakultetet å ha god og omfattende kontakt med den anvendte instituttsektoren. For numerikerne og særlig databehandlerne har særlig samarbeidet med NR og SI betydd mye. Simula, som ble lansert i 1967, ble utviklet gjennom dette samarbeidet. Det første professorat II innen databehandling ble besatt av Sverre Spurkland, hadde sin hovedstilling som forskningssjef ved NR. Det samme hadde som nevnt Kristen Nygaard. Ved utgangen av 1973 hadde 11 av de 44 aktive hovedfagsstudentene som var i gang med sin hovedoppgave, veiledning ved NR. Kybernetikerne hadde sine bredeste kontaktflater til SI og Kjeller-instituttene. Tilsettingen av Yngvar Lundh representerte således en naturlig forlengelse av et tradisjonelt samarbeid.

Samarbeidet mellom instituttet og NR/SI er omfattende, særlig innen forskning og forskerutdanning. Mengden av veiledede hovedfagsstudenter ved NR og SI har avtatt litt de siste årene, men dette er langt på veg blitt kompensert ved at aktiviteten ved Kjellerinstituttene har økt. Instituttet er fakultetets fremste bruker av UNIK samarbeidet. Dette samarbeidet ble formelt organisert i 1987, som et samarbeid mellom matnat-fakultetet og FFI, Televerkets forskningsinstitutt og Institutt for energiteknikk. I dag er det særlig faggruppene for digitalteknikk og matematisk modellering som har nytte av samarbeidet, og de siste år har nær 20\% av de uteksaminerte cand.scient.-kandidatene ved instituttet hatt sin veiledning ved UNIK.

\subsection{Nye satsningsområder - doktorgradsutdanning}
Et satsningsområde som er av relativt ny karakter, i alle fall i noe betydelig omfang, er den organiserte forskerutdanningen. Antallet doktorgradsstudenter har de siste år økt i et akselererende tempo. Ved inngangen til jubileumsåret har instituttet 66 aktive doktorgradsstudenter. Av disse er 19 ansatt ved universitetet og 15 har finansiering via eksterne midler, i hovedsak via NFR. Antallet uteksaminerte doktorgradskandidater pr år henger relativt sett foreløpig noe etter. Doktorgradsstipendiatene representerer en vesentlig styrking av den totale forskningsinnsatsen ved instituttet. Også innen undervisning og veiledning representerer de en betydelig ressurs. Samtidig er de en belastning på instituttets svært anstrengte rom situasjon, og har, sammen med økningen i tallet på ansatte for øvrig, ført til at hovedfagsstudentene langt på veg er i ferd med å bli trengt ut av Informatikkbygget. Doktorgradsveiledningen legger også beslag på en betydelig del av veiledningskapasiteten til det fast ansatte vitenskapelige personalet.

En ny stillingskategori er også post.doc. ansatte. Instituttet har ikke selv slike stillinger. De 4 personene som sitter i slike stillinger er således alle finansiert av NFR.

\subsection{Situasjonen i 1994}
Ved inngangen til jubileumsåret er antallet ansatte 116, inklusive bistillinger og forskningsrådsfinansierte stipendiater. Fra å være en Benjamin ved fakultetet er instituttet vokst seg opp til å bli et av de middels store instituttene, også når det gjelder antall ansatte. Studenttallet er fortsatt blant de høyeste på fakultetet, som det alltid har vært. I dag er det særlig fagretningene for digitalteknikk og systemarbeid som er presset. Ikke minst fordi det er disse fagretningene som også har det laveste antallet faste stillinger.

Fortsatt er forholdstallet mellom lærere og studenter betydelig vanskeligere ved Ifi enn ved de andre matnat-instituttene. Ut fra de normer fakultetet regner ut sine forholdstall etter, er det 4 ganger så mange studenter pr. lærer ved IFI enn ved noe annet institutt. Men kanskje nettopp fordi forholdstallet mellom lærere og studenter ved instituttet har vært og er så dårlig, er instituttet kjennetegnet av et svært nært forhold mellom lærere og studenter. Det er ikke hverken plass eller tid til isolasjon og ærbødig avstand mellom studenter, vitenskapelig personale og teknisk-administrativt personale.

Om det er det tette og gode arbeidsmiljøet som er den eneste årsaken skal være usagt, men det er et faktum av instituttet skårer meget høyt på statistikken over uteksaminerte hovedfagskandidater pr. vitenskapelig ansatt, trolig høyest ved universitetet. Antallet uteksaminerte cand. scient. har de siste årene har ligget på 80-90 kandidater pr år. Instituttet sto i 1992 for 5,6\% av universitetets samlede produksjon av høyere grads kandidater, mens det bare disponerte 1,8\% av stillingsmassen.

Det har blitt reist spørsmålstegn om hvorvidt denne studentsentrerte aktiviteten på undervisningssiden fører til at forskningssiden blir dårligere ivaretatt. I årsplanen for jubileumsåret har instituttet som sin høyest prioriterte oppgave nettopp å stimulere til at de vitenskapelige ansatte setter av tid til forskning, bl.a. ved å øke publiseringsfrekvensen - uten at dette skal gå ut over innsatsen i undervisning og veiledning.

\subsection{Nøkkeltall}
En del nøkkeltall kan illustrere utviklingen i antall ansatte og studenter i perioden 1970 til i dag.

%TODO tabell, stillinger med fotnoter/kommentarer

\section{Kristen Nygaards teknologiske konstruksjon av arbeidsplassdemokrati}
%TODO plasseres sammen med OJD-artikkel

av Egil Øvrelid
(originalt trykket i studenttidsskriften Index 5. mai 2016)
% https://issuu.com/ifiindex/docs/utgave_2_2016_web - side 8-11

Kristen Nygaard døde i 2002, 76 år gammel, men arven etter hans arbeid lever fortsatt. Den kan ses blant annet i det brede fokuset innføringen av kliniske IT-systemer har i dagens Helsevesen. Det store programmet «Digital fornying» i Helse Sør-Øst, som har en prislapp på 6 milliarder i perioden 2013-2020, handler blant annet om utvikling og standardisering av kliniske applikasjoner. Kravspesifikasjonene som sendes ut på anbud er utarbeidet i tett samspill med en rekke klinikergrupper. Brukernes aktive deltagelse er en selvfølge. Denne formen for deltagende utvikling har sin kilde i det vi kan kalle den «skandinaviske modellen for systemutvikling» som har sitt opphav i Nygaards og Ole-Johan Dahls arbeid etter krigen og Nygaards og Olav Terje Bergos Jern- og Metallprosjekt sammen med Fagforeningen på begynnelsen av 1970-tallet. Arbeidet ledet allerede tidlig i 1970-årene til at det ble inngått dataavtaler og oppnevnt datatillitsvalgte i arbeidslivet. 

\subsection{Operasjonsanalysen}
Kristen Nygaards karriere startet på Forsvarets forskningsinstitutt (FFI) rett etter krigen. Han jobbet med prosjekter knyttet til modernisering av Forsvaret, som var tett knyttet til gjenoppbygningen av landet og industrien etter 5 år under okkupasjon. Utover 1940- og 50-tallet var Norge langt fremme både innen kjernekraft og produksjon av militærteknologi, og produksjonsmodellene herfra ble gjeldende også for annen industri. Nygaards engasjement og kunnskap vokste frem her, og det var flere elementer som påvirket hans virke frem til 1975.

Først Operasjonsanalysen som Nygaard brukte aktivt i sitt arbeid både på FFI og Norsk Regnesentral. Operasjonsanalysen (OA) er en matematisk kvantifiserbar vitenskap som anvendes for å finne det mest effektive samspillet mellom militære teknologier som fly og militært materiell i krigføringen. Operasjonsanalysen viste seg svært effektiv under andre verdenskrig. Simulering ble brukt for å modellere kommunikasjonsstrømmen mellom komponentene i den militære teknologien, og Nygaard videreutviklet operasjonsanalysens virkeområde ved å integrere soldatene tettere inn i eksperimentene, samtidig som han deltok selv. Systemanalyse er en annen retning innen OA, men dens fokus på økonomi forskjøv beregningstyngden over på en ledelsesdiskurs som handlet om å velge det mest lønnsomme, ikke lenger det vitenskapelig riktige. Kristen Nygaard kunne ikke aksepte dette. Det skiftende fokuset fra grunnivået der soldatene og teknologien opererer til Systemanalyse der økonomi og ledelse dominerer ble for mye å svelge for Nygaard. Han sa derfor opp hos FFI og gikk til Norsk Regnesentral i 1960.

\subsection{Arbeiderne i fokus}
Deretter er Aksjonsforskningen til Tavistock-skolen tilegnet fra gruvene i Nord-England på 1950 tallet en viktig inspirasjonskilde i Nygaards arbeid. Tavistock ble opprettet like etter første verdenskrig, og ble utvidet med «Institute of Human relations» i 1947, der samfunns- og arbeidsforhold sto sentralt. Forskningen til Tavistock gikk ut på å dokumentere problemene som oppsto i overgangen fra en autonom modell med selvstyrte små team, til en omfattende oppdeling av arbeidet i ulike prosesser, og med flere skift. Det viste seg at effektiviteten gikk ned, og at arbeiderne tok mindre ansvar for helheten i arbeidet. Den sosiotekniske systemforskningen har sitt opphav her, men Tavistocks «idealtype» med små selvstyrte team som ivaretok både nærsamfunn, arbeidet og arbeiderne, skalerte dårlig i den nye samfunnsøkonomien basert på stordrift, masseproduksjon og spesialisering. Dette ble inspirasjon for et tilsvarende prosjekt i Norge. Dette ble støttet av NAF, Jern- og Metall og den norske Stat, og det norske arbeidslivet ble sett på som spesielt egnet for slike forsøk. Målet med det norske prosjektet var å «forbedre betingelsene for personlig medvirkning i den konkrete arbeidssituasjonen med sikte på å utløse menneskelige ressurser.» Gjennom rotering på arbeidsoppgaver skulle arbeiderne få sterkere eierskap og friere utfoldelse på arbeidsplassen. Prinsippene fra prosessene i gruvene i Tavistock ble videreført, men tilpasset dem til den moderne industrien. Imidlertid var den strategiske og organisatoriske planleggingen i bedriften fortsatt i ledelsens vold.

Dette norske prosjektet var bakgrunnen for at Nygard og Bergo startet et prosjekt sammen med Jan Balstad fra Jern- og Metall. De hadde som eksplisitt forutsetning at samarbeidsprosjektene til Thorsrud og Emery ikke gikk langt nok i prosessforbedringen, «at medvirkningen skjedde på et for sent tidspunkt i teknologiutviklingen, og at all kunnsakpsutvikling skjedde på ledelsens og forskernes premisser». Nygaard var krystallklar: Arbeidstakerne måtte gis dypere i innsikt i bedriftsledelse og styring i tillegg til produksjon.

\subsection{SIMULA}
Kristen Nygaard var først og fremst informatiker og programmerer, og ble etter hvert sterkt drevet av objektorientert tenkning. Gjennom erfaringene med simulering fra krigen og hvordan ulike komponenter (inkludert soldaten) kan forstås som objekter i systemet, lagde Nygaard og Dahl SIMULA, verdens første objektorienterte programmeringsspråk. SIMULA ble et pedagogisk språk som muliggjorde en helhetlig systemutviklingsprosess der arbeiderne kunne delta fra spesifikasjon og planlegging og helt til innføringen av systemet i organisasjonen. I SIMULA fikk dataelementene egenskaper, og ble således dynamiske representanter i systemet for verden utenfor. Den grunnleggende endringen besto i at arbeidernes systemverden ble satt i sentrum på en helt annen måte. Jern og Metall-prosjektet tok inn i seg alle disse strømningene i en kraftfull cocktail som skulle skape en brukerstyrt teknologisk sfære som dynamisk kunne tilpasses og brukes i enhver industrisammenheng. 

Vi har sett noe av løsningen til Nygaard, men hvilket samfunnsproblem var det han forsøkte å løse? 


Det moderne industrisystemet som vokste frem etter andre verdenskrig var basert på sterk statlig deltagelse i industri- og samfunnsbyggingen. Det var i utgangspunktet lagt opp demokratisk, men visse krefter trakk det bort fra fokus på arbeidstakerens teknologiske utvikling, og isteden mot økonomisk eller teknokratisk optimalisering. Industrisystemet var meget komplekst, basert på teknologisk og økonomisk utvikling, og omfattende kunnskap var nødvendig for å styre det. Universitetene ble den sentrale institusjonen, og utdanning den sentrale faktoren for å bli politiker og industrileder. Den politiske og industrielle ledelsen var basert på utdanning og kunnskap fra universitetene, samtidig som arbeiderklassen havnet i bakleksa. Skillet mellom utdannede og ikke-utdannede truet balansen og den demokratiske deltagelsen i samfunnet. Et fundamentalt problem med den ledelsesorienterte og tidvis teknokratiske diskursen var at den førte til «dekvalifisering» av arbeidstakeren, der de som jobbet på gulvet verken hadde kunnskap eller forutsetninger til å forstå hvordan systemet fungerte. Arbeidstakeren havnet i teknologiens vold, og ble fratatt alle menneskelige egenskaper i arbeidsutføringen.

Her gir objektorienteringen arbeidstakerne et språk de kunne anvende til å kommunisere de sosiale perspektivene som skulle oversettes til teknologiske interaksjoner i systemet. Når arbeiderne selv er med å bestemme egenskapene til sine objekter i systemet påvirker de direkte styringen av systemet, fordi objektene er deler av et system som danner grunnlag for de avgjørelser som fattes av de som styrer. SIMULA skulle således bidra til en «rekvalifisering» av menneskelige egenskaper. En forflytning av industriell kapasitet til informasjonsteknologisk innsikt. 

Kristen Nygaard ville nok vært ambivalent til de store Helse-IT prosjektene som pågår i Norge nå der sentralisering går foran desentralisering, der økonomiske perspektiver settes i forgrunnen, og der den kliniske ekspertise ofte må konkurrere mot ledelse og økonomi. På den andre side handler også det moderne helsevesenet om smarte pasienter. Et system der pasienten blir stående litt med «lua i hånda», prisgitt uoversiktlige maktsystemer, er ikke lenger et moderne system. Akkurat som industriarbeidere hadde påvirkningskraft i 1970-årene burde pasientene også være objekter med påvirkningskraft i 2020-årene, og fremover. Åpenhet, oversiktlighet, deltagelse og eierskap er til alle pasienters beste, ikke bare de mest ressurssterke.

\end{document}